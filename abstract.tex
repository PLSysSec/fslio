\begin{abstract}
  Flow-sensitive analysis for information-flow control (IFC) allows data
  structures to have mutable security labels, i.e., labels that can change over
  the course of the computation.
  %
  This feature is often used to boost the permissiveness of the IFC monitor, by
  rejecting fewer programs, and to reduce the burden of explicit label
  annotations.
  %
  However, when added naively, in a purely dynamic setting, mutable labels can
  expose a high bandwidth covert channel.
  %
  In this work, we present an extension for LIO---a language-based
  floating-label system---that safely handles flow-sensitive references.
  %
  The key insight to safely manipulating the label of a reference is to not
  only consider the label on the data stored in the reference, i.e., the
  reference label, but also the \emph{label on the reference label} itself.
  %
  Taking this into consideration, we provide an \emph{upgrade} primitive that
  can be used to change the label of a reference in a safe manner.
  %
  To eliminate the burden of determining when a reference should be upgraded,
  we additionally provide a mechanism for automatic upgrades.
  %
  Our approach naturally extends to a concurrent setting, not previously
  considered by dynamic flow-sensitive systems.
  %
  For both our sequential and concurrent calculi, we prove non-interference by
  embedding the flow-sensitive system into the flow-insensitive LIO calculus,
  a surprising result on its own.
\end{abstract}
