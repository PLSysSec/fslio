\begin{abstract}
  Flow-sensitive analysis for information-flow control (IFC) allows data
  structures to have mutable security \emph{labels}, i.e., labels that can
  change over the course of the computation. This feature is beneficial for
  several reasons: it often boosts permissiveness, reduces the burden of
  explicit annotations, and enables reuse of single-threaded data structures.
  However, in a purely dynamic setting, mutable labels can expose a covert
  channel capable of leaking secrets at a high rate.
  %
  We present an extension for LIO---a language-level floating-label
  system--that safely handles flow-sensitive references.
  %
  The key insight to safely manipulating the label of a reference is to not
  only consider the label on the data stored in the reference, i.e, the
  reference label, but also the \emph{label on the reference label} itself.
  %
  Taking this into consideration, we provide an \emph{upgrade} primitive that
  can be used to change the label of a reference, when deemed safe.
  %
  To eliminate the burden of determining when it is safe to upgrade a
  reference, we additionally provide a mechanism for the automatic upgrades.
  %
  Our approach naturally extends to a concurrent setting, not previously
  considered by flow-sensitive systems.
  %
  For both our sequential and concurrent calculi we prove non-interference by
  embedding the flow-sensitive system into the flow-insensitive LIO calculus;
  the embedding itself is a surprising result.
\end{abstract}
