\begin{abstract}
  Flow-sensitive analysis for information-flow control (IFC) allows
  data structures to have mutable security annotations,
  i.e. annotations that can change over the course of the
  computation. This feature is beneficial for several reasons: it
  often boosts permissiveness, reduces the burden of explicit
  annotations, and enables reuse of single-threaded data
  structures. In a purely dynamic setting, although attractive,
  mutable labels can expose a covert channel capable of leaking
  secrets at a high rate.
  %Existing language-based solutions cover sequential languages, where it
  %is forbidden to either change the labels of certain flow-sensitive variables
  %or branch on them.  
  We present an extension for LIO---a floating-label system based on
  ideas from the operating system domain---capable of securely
  handling flow-sensitive references. The key insight is considering
  labels as being composed of two elements: a label indicating the
  confidentiality of data, and another label, called \emph{the label
    on the label}, which describes the confidentiality of the data's
  label itself. Within this model, our approach allows label changes
  subject to the {\nsu} policy. Additionally, we introduce operations
  to upgrade labels in safe situations and a mechanism for the
  automatic placement of these operations. As a surprising result, we
  prove that {\nsu} and upgrade operations can be encoded by using
  flow-insensitive concepts---for that, we leverage the ability of LIO
  to have nested labeled values.
%the structure of the security lattice (e.g. no relabeling of
%secrets as public data) and branches on altered flow-sensitive variables.
% Differently from mainstream IFC tools, {\LIO} allows to query the security
% labels associated to entities. 
% This feature becomes particularly important to
% secure in presence of flow-sensitive entities.
  In addition, we extend our solution for concurrent systems almost naturally,
  except for a subtle change to avoid the termination channel. 
% We show, by an attack, that \emph{{\nsu}} is the most suitable
% discipline for label changes where execution is stopped only based on threads’
% local data---an essential requirement to obtain performance and scalability.
\end{abstract}

% Local Variables:
% TeX-master: "main.ltx"
% TeX-command-default: "Make"
% End:



 


