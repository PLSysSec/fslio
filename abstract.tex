\begin{abstract}
Flow-sensitive analysis for information-flow control (IFC) allow to change the
security annotations provided to data structures (e.g. variables). This feature
is beneficial for several reasons: it often boosts permissiveness, reduce the
burden of explicit annotations, and enables the reuse of single-threaded data
structures. In a purely dynamic setting, although attractive, changes on labels
can become a covert-channel capable to leak secrets at a high rate. Existing
language-based solutions cover sequential languages, where it is forbidden to
either change the labels of certain flow-sensitive variables or branch on them.
We present an extension for LIO---a floating-label system based on ideas from
the operating system domain---which is capable to safely handle flow-sensitive
and flow-insensitive mutable references. Our approach allows both arbitrary
label changes subject to the structure of the security lattice (e.g. no
relabeling of secrets as public data) and branches on altered flow-sensitive
variables.  Differently from mainstream IFC tools, LIO allows to query the
security labels associated to entities. This feature becomes particularly
important to secure in presence of flow-sensitivity. In addition, we explore the
solution space for secure label changes in presence of concurrency. We present
an attack which suggest that \emph{{\nsu}} is the most permissive
discipline for label changes in the concurrent version of LIO, where
execution is stopped only based on threads’ local data---an essential
requirement to obtain performance and scalability.
\end{abstract}






  




% Local Variables:
% TeX-master: "main.ltx"
% TeX-command-default: "Make"
% End:
 


