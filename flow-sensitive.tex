\section{Flow-sensitivity extensions}
\label{sec:flow-sensitive}

%\concept{a more-LIO focused example for why FS}
Unfortunately, the flow-insensitive references described in the previous section
are somewhat inflexible.
%
Consider, for example, an application that uses a reference as a log.
%
Since the log may contain sensitive information, it is important that the
reference be labeled.
%
Equally important is to be able to read the log at any point in the program to,
for instance, save it to a file.
%
Although labeling the reference with the top element in the security
lattice ($\top$) would always allow writes to the log, and |toLabeled|
can be used to read the log and then write it to a file, this is
unsatisfactory: it assumes the existence of a top element, which in
some practical IFC systems, including HiStar~\cite{zeldovich:histar}
and Hails~\cite{giffin:hails}, does not exist.
%
Moreover, it almost always over-approximates the sensitivity of the
log.
%
Hence, for example, a computation that never reads sensitive data, yet
wishes to read the log content as to send error message to a user over
the network (e.g., as done in a web application) cannot do so---LIO
prevents the computation from read the log, thereby tainting itself
$\top$, and subsequently writing to the network.\footnote{
  Here, as in most IFC systems, we assume the network is public.
}
%
It is clear that even for such a simple use case, having references
with labels that vary according to the sensitivity of what is stored
in the reference is useful.

%\concept{naive approach doesn't work}
% Ale: I don't like implicit since it might be associated to implicit flows
However, naively implementing flow-sensitive references can effectively
introduce label changes as a covert channel. 
%similar to other language-based system to LIO.
%
Suppose that we allow for the label of a reference to be raised to the
current label at the time of the |writeRef space|.
%
So, for example, if the label of our log reference is |low| and the
computation has read sensitive data (such that the current label is
|H|), subsequently writing to the log will raises the label of the
reference to |H|.
%
Unfortunately, while this may appear safe, as previously show
in~\cite{Austin:Flanagan:PLAS09,Russo:2010, Austin:Flanagan:PLAS10},
the approach is unsound.

\begin{figure}[t]
\small
\begin{code}
leakRef :: LIORefTCB space Bool -> LIO Bool
leakRef href = do
  tmp   <- newRef space low ()
  toLabeled high $ do  h <- readRef space href
                       when h $ writeRef space tmp ()
  return $ labelOfR space tmp == high
\end{code}
\cut{$}
\caption{Attack on LIO with naive flow-sensitive reference extension. We omit
subscripts for clarity.  \label{fig:fs-attack}}
\end{figure}

The code fragment in Figure~\ref{fig:fs-attack} defined a function,
|leakRef|, that can be used to leak the contents of a reference by
leveraging the newly introduced covert channel: the label of references.
%
(In this and future examples we use function |when| to denote an |if|
statement without the |else| branch and |($)| as lightweight notation
for function application, i.e., |f $ x| is the same as |f (x)|.)
%
To illustrate an attack, suppose that the current label is public
(|low|) and |leakRef| is called with a secret (|high|) reference
(|href|).
%
|leakRef| first creates a public reference |tmp| and, then, within the
|toLabeled| block---which is used to ensure that the current label remains
|low|---the label of this reference is changed to |H| if the secret stored in
|href| is |true|, and left intact (|L|) if the secret is |false|.
%
Finally, by simply inspecting the label of the reference the value stored in
|href| is revealed.\footnote{%
The use of |labelOfR space| is not fundamental to this attack and in
Appendix~\ref{sec:app:fs-attack2} we show an alternative attack that
does not rely on such label inspection.
}

%\concept{KEY INSIGHT:}
Fundamentally, the label protecting the \emph{label} of an object,
such as a reference or labeled value, is the current label |lcurr| at
the time of creation.
%
Hence, to modify the label of the object within some context (e.g., |toLabeled|
block) wherein the current label is |lcurr'|, \emph{it must be the case that} |lcurr'
canFlowTo lcurr|, i.e., we must be able to write data at sensitivity level
|lcurr'| into an entity---the label of the object---labeled |lcurr|.
%
This restriction is especially important if |lcurr canFlowToStrict
lcurr'| and we can restore the current label from |lcurr'| to |lcurr|, 
since a leak would then be observable within the program itself.
%
In the case where the label of the object is immutable, as is the case for
flow-insensitive references (and labeled valued), this is not a concern: even if
the current label is raised to |lcurr'| and then restored to |lcurr|, we do not
learn any information more sensitive than |lcurr|---the label of the label at the
time of creation---by inspecting the label of the reference (or
value): the label has not changed!
%

\begin{figure} % semantics
\small
\begin{code}
v    ::= cdots  | LIORefTCB S t
t    ::= cdots  | upgradeR t t | diverge
Ep   ::= cdots  | upgradeR Ep t | upgradeR v Ep
\end{code}

\begin{mathpar}
\inferrule[newRef-|S|]
{ |c = (lcurr, mI, mS)|\\
  |lcurr canFlowTo l|\\
  |fresh a|\\
  |mS' = mS[a mapsto LabeledTCB lcurr (LabeledTCB l t)]|\\
  |c' = (lcurr, mI, mS')|
}
{ |conf c (E[newRef S l t]) --> conf c' (E[return (LIORefTCB S a)])| }

\and

\inferrule[readRef-|S|]
{ |c = (lcurr, mI, mS)| \\
  |mS(a) = LabeledTCB l (LabeledTCB l' t)|\\
  |l'' = l lub l'|
}
{ |conf c (E[readRef S (LIORefTCB S a)]) --> conf c (E[unlabel (LabeledTCB l'' t)])| }

\and

\inferrule[writeRef-|S|]
{ |c = (lcurr, mI mS)|\\
  |mS(a) = LabeledTCB l (LabeledTCB l' t')| \\
  |lcurr canFlowTo (l lub l')|\\
  |mS' = mS[a mapsto LabeledTCB l (LabeledTCB l' t)]|\\
  |c' = (lcurr, mI, mS')|
}
{ |conf c (E[writeRef S (LIORefTCB S a) t]) --> conf c' (E[return ()])| }

\and

\inferrule[writeRef-|S|-fail]
{ |c = (lcurr, mI mS)|\\
  |mS(a) = LabeledTCB l (LabeledTCB l' t')| \\
  |lcurr cantFlowTo (l lub l')|
}
{ |conf c (E[writeRef S (LIORefTCB S a) t]) --> conf c' (E[unlabel (LabeledTCB l diverge)])| }

\and

\inferrule[labelOf-|S|]
{ |c = (lcurr, mI, mS)|\\
  |mS(a) = LabeledTCB l (LabeledTCB l' t)|
}
{ |conf c (E[labelOfR S (LIORefTCB S a)]) --> conf c (E[unlabel (LabeledTCB l l')])|}

\and

\inferrule[upgradeRef]
{ |c = (lcurr, mI, mS)|\\
  |mS(a) = LabeledTCB l v|\\
  |lcurr canFlowTo l| \\
  |conf c (upgrade v l') ==> conf c (LIOTCB v')|\\
  |mS' = mS [ a mapsto LabeledTCB l v']|\\
  |c' = (lcurr, mI, mS')|
}
{ |conf c (E[upgradeR (LIORefTCB S a) l']) --> conf c' (E[return ()])| }
\end{mathpar}
\caption{\liofs{}: \lio{} with flow-sensitive references.\label{fig:sos:fs}}
\end{figure}


%\concept{\liofs{}}
Thus, to extend LIO with flow-sensitive references, we must account for the label
on the label of the reference at the time of creation, |lcurr|.
%
(This label is, however, immutable.)
%
In turn, when changing the label of the reference, we must ensure that no data
from the context at the time of the change, whose label is |lcurr'|, is leaked
into the label of the reference by ensuring that |lcurr' canFlowTo lcurr|,
i.e., we can write data labeled |lcurr'| into the label that is labeled |lcurr|.
 
Formally, we extend the \lio{} syntax and reduction rules as shown in
Figure~\ref{fig:sos:fs}; we call this calculus \liofs{}.
%
When creating a flow-sensitive reference, |newRef S l t| creates a labeled
value that guards |t| with label |l| (|LabeledTCB l t|).
%
However, since we wish to allow programmers to modify the label |l| of the
reference, we additionally store the label on |l|, i.e., |lcurr|, by simply
labeling the already-guarded term (|mS' = mS[a mapsto LabeledTCB lcurr
(LabeledTCB l t)]|), as shown in rule~\ruleref{newRef-|S|} .
%
Primitive |newRef S| returns a |LIORefTCB S a| which simply encapsulates the fresh
reference address where the doubly-labeled term is stored.
%
Different from the constructor |LIORefTCB I|,  the constructor |LIORefTCB S|
does not encapsulate the label of the reference. 
%
This is precisely because the label of a flow-sensitive reference is 
mutable and must be looked up in the store.
%
As given by rule~\ruleref{labelOf-|S|}, |labelOfR S| returns the label of the
reference after raising the current label (with |unlabel|) to account for the
fact that the label of the reference |l'| is a value at sensitivity level |l|,
i.e., we raise the current label to the join of the current label and the label
on the label.

The rule for reading flow-sensitive references is standard.
%
As given by \ruleref{readREf-fs}, |readRef S| simply raises the
current label to join of the reference label and label on the
reference label (|l lub l'|) and returns the protected value.
%
This reflects the fact that the computation is observing both data at
level |l| (the label on the reference) and |l'| (the actual term).


\begin{wrapfigure}{r}{0.5\columnwidth}
\vspace{-15pt}
\begin{code}
do  r <- newRef S high ()
    readRef S r
    writeRef S r ()
\end{code}
\vspace{-15pt}
\caption{\label{fig:permissive} Permissiveness.}
\end{wrapfigure}
%
The rule for writing flow-sensitive references deserves more
attention.
%
First, |writeRef S| ensures that the current computation can write to
the reference by checking that |lcurr canFlowTo (l lub l')|.
%
We impose this condition instead of the two conditions |lcurr
canFlowTo l| and |lcurr canFlowTo l'|---which respectively check that
the current computation can modify both, the label of the reference,
and the reference itself---since it is more permissive, yet still
safe.
%
When imposing the two conditions independently, certain programs, such
as the one given in Figure~\ref{fig:permissive}, would fail.
%
In this program, we first create a flow-sensitive reference labeled
|high| when the current label is |low| (and thus the label on |high|
is |low|).
%
Then, we raise the label by reading from the reference.
%
Finally, we attempt to write the to the reference.
%
Under our semantics, this program behaves as expected; however, when
imposing the two condition independently, the write fails---the
current label does not flow to the label on the label of the
reference.
 
Another peculiarity with |writeRef S| is the case when the current label does not
flow to the join of the reference label, i.e., |lcurr cantFlowTo l lub l'|, and
thus the write should not be allowed.
%
If the semantics simply got stuck, the current label (at the point of
the stuck term) would
not reflect the fact that the success of applying such rule depends on the
label |l'|, which is itself protected by |l|.
%
Indeed, this might lead to information leaks and we thus we provide an
explicit rule, \ruleref{writeref-fs-fail}, for this failure case that
first raises the current label (through |unlabel|) to |l| and then
diverges; in the rule, |diverge| represents a divergent term for which
we do not provide a reduction rule.

% there is a termination leak which reveals
% information about the possible changes on label |l'| (itself protected by |l|).
% Observe that those changes could have occurred in a more sensitive context than
% |lcurr|. For sequential LIO, this is not a problem since the security guarantees
% ignore termination leaks~\cite{stefan:lio}. However, in a concurrent setting,
% such leaks are a concern~\cite{stefan:addressing-covert}. Aiming to apply our
% flow-sensitive approach for concurrency almost without modifications (see
% Section~\ref{sec:conc}), we introduce a failure rule for |writeRef S|
% which
%does not reflect the fact
%that the rule relies, and thus observes, the label on the reference |l'| which
%is itself protected by |l|
%
%The effects of this rule in a sequential settings is analogous to getting
%stuck\footnote{This is not true if we consider exceptions, a feature 
%out of the scope of this paper}. 
Finally, we note that |writeRef S| does not modify the label of the reference.
%
This is, in part, because we wish to keep the difference between
flow-insensitive and flow-sensitive references as small as possible.
%
Instead, we provide |upgradeR| precisely for this purpose; this primitive is
used to raise the label of the reference.
%
Rule \ruleref{upgradeRef} is straight forward---it simply ensures that
the current computation can modify the label of the reference by
checking that the current label flows to the label on the label
(|lcurr canFlowTo l|).




\subsection{Automatic upgrades}
\label{sec:flow-sensitive:auto}

\begin{figure} % auto-fs
\small
\begin{mathpar}
\inferrule[upgradeStore]
{|c = (lcurr, mI, mS)|\\
 |mS = {a1 mapsto v1, ..., a_n mapsto v_n}|\\
 |ti = upgradeR (LIORefTCB S a_i) l, i = 1, ..., n|
}
{|conf c (E[upgradeM l]) --> conf c (E[t1 >> ... >> t_n])|}
\and
\inferrule[unlabel-au]
{ |c = (lcurr, mI, mS)|\\
  |lcurr' = lcurr lub l|\\
  |conf c (upgradeM lcurr') ==> conf (lcurr, mI, mS') (LIOTCB ())|\\
  |c' = (lcurr', mI, mS')|
}
{
|conf c (E[unlabel (LabeledTCB l t)]) --> conf c' (E[return t])|
}
\end{mathpar}
\caption{\lioafs{}: Extending \liofs{} with auto-upgrades.\label{fig:sos:afs}}
\end{figure}

\begin{figure} % auto-fs
\small
\begin{code}
v    ::= cdots  | set (v, ...) | nil
t    ::= cdots  | set (t, ...) | withRefs t t
tau  ::= cdots  | set (tau, ...)
Ep   ::= cdots  | set (Ep, t, ...) | set(v, Ep, t, ...) | withRefs Ep t 

addrs (nil)                                       @= emptyset
addrs (set(LIORefTCB S a1, LIORefTCB S a2, ...))  @= {a1, a2, ...}
\end{code}
\begin{mathpar}
\inferrule[withRefs-Ctx]
{|c = (lcurr, mI, mS)|\\
 |mS' = {a mapsto mS(a) inlinesep a element dom mS and (addrs v) }|\\
 |conf (lcurr, mI, mS') (E[t]) --> conf (lcurr', mI', mS'') (E[t'])|\\
 |c'' = (lcurr', mI', mS'' merge mS)|\\
 |v' = refs(dom mS'')|
}
{|conf c (E[withRefs v t]) --> conf c'' (E[withRefs v' t']|}
\and
\inferrule[withRefs-Done]
{ }
{|conf c (E[withRefs v v']) --> conf c (E[v']|}
\and
\inferrule[Type-withRef]
{ |D' = {a mapsto D(a) inlinesep a element dom D and (addrs v) }|\\
  |D', G :- v :  set (LIORef S tau1, ...)|\\
  |D', G :- t :  LIO tau|
}
{ |D, G :- withRefs v t : LIO tau| }
\end{mathpar}
\caption{Extending \liofs{} and \lioafs{} with |withRefs|.\label{fig:sos:withRefs}}
\end{figure}

%\concept{upgrade is burdensome}
We can use \liofs{} to implement various applications that rely on
flow-sensitive references, even those that rely on policies such as the
popular no-sensitive upgrades~\cite{Austin:Flanagan:PLAS09}.
%
%\hl{Ale: to do}
(In Section~\ref{sec:related}, we describe the encoding of a policy that is
similar to, but more permissive than, no-sensitive upgrades.)
%
Using \liofs{}, we can also safely implement our logging application using a
flow-sensitive reference.
%
Unfortunately, our system (and others like it) requires that we insert
|upgrade|s before we raise the current label so that it is possible to 
write references in a more-sensitive context, e.g., to modify a public reference
(with label on the label |L|) after reading a secret. 
%
In the case of the logging example, we would need to upgrade the label
before reading any sensitive data, if we later which wish to write to
the log.

%\concept{auto upgrade}
Inspired by~\citep{Hedin13}, we thus provide an extension to \liofs{} that can
be used to automatically upgrade references.
%
This extension, called \lioafs{}, is given in Figure~\ref{fig:sos:afs}.
%
Intuitively, whenever the current label is about to be raised, we first upgrade
all the references in the |mS| store %, with |upgradeM|, in the store 
and then raise the current label.
%
Rule \ruleref{upgradeStore} upgrades every reference in the flow-sensitive store
|mS| by executing |t_1 >> t_2 >> cdots >> t_n|, where 
|ti = upgradeR (LIORefTCB S a_i) l|. Term |t >> t'| is
similar to bind except that it discards the result produced by
|t|.
%\footnote{Formally, |t >> t'| is defined as |t >>= \x.t'| where |x| does not
%  appear free in |t'|}
%
Since |unlabel| is the only function that raises the current label, we
augment the \ruleref{unlabel} rule with \ruleref{unlabel-au}, given in
Figure~\ref{fig:sos:afs}.
%
This ensures that as the computation progresses it does not ``lose''
write access to its references.
%
Returning to our logging example, with auto-upgrades, the reference used as the
log never needs to be explicitly upgraded and can always be written to---an
interface expected of a log.

%\concept{store creep}
Recall, however, that |toLabeled| is used to avoid label creep by
allowing code to only temporarily raise the current label.
%
Unfortunately, with auto-upgrades, when the current label gets raised 
within a |toLabeled| block, the upgrades of the flow-sensitive references remain even
after the current label is restored.
%
Thus, reading from any flow-sensitive reference after the |toLabeled|
block will raise the current label to (at least) the current label at the end
of the |toLabeled| block (since all references are upgraded every time that 
the current label gets raised).
%
This can be used to carry out a \emph{poison pill}-like
attack~\cite{Breeze}, wherein the (usually untrusted) computation
executing within the |toLabeled| block will render the outer computation useless
by imposing label creep.
%
(We note that this attack is possible in \liofs{} without the auto-upgrade, but
requires the attacker to manually insert all the upgrades.)

%\concept{withRefs}
To address this issue, we extend \liofs{} (and thus \lioafs{}) with |withRefs v
t| which takes a bag (strict heterogeneous list) |v| of references and a computation |t|, and executes |t|
in a configuration where the flow-sensitive reference store only contains the
subset of references |v|.
%
%
This extension and type rule \ruleref{Type-withRef}, which ensures that a term
cannot access a reference outside its store, are shown in
Figure~\ref{fig:sos:withRefs}. 
%
A bag is either empty |nil|, or it may contain a set of references
of (potentially) distinct types |set (v,..)|.
%
Rules \ruleref{withRefs-Ctx} and \ruleref{withRefs-Done} precisely define the
semantics of this new primitive, where meta function |addrs| converts a bag of
references to a set of their corresponding addresses, |refs| performs the
inverse conversion, and $\;|merge|\;$ is used to merge the stores, giving preferences
to the left-hand-side store, i.e., when there is a discrepancy on a stored
value between both stores, it chooses the one appearing on the left-hand-side.
%
We note that \ruleref{withRefs-Ctx} is triggered until the term under
evaluation is reduced to a value, at which point \ruleref{withRefs-Done} is
triggered, returning said value; we specify this big-step rule in terms of
small-steps to facilitate the formalization of our concurrent calculus (see
Section~\ref{sec:conc}).
% 
Aside from the modeling of bags, the |withRefs| primitive is
straightforward and mostly standard;
indeed, the programming paradigm is similar to that already present in
some mainstream languages (e.g., C++'s lambda closures require the
programmer to specify the captured references).
%
Lastly, we note that the poison pill attack can now be addressed by
simply wrapping |toLabeled| with |withRefs|, which prevents
(untrusted) code within the |toLabeled| block from upgrading arbitrary
references. 


%%%%%%%%%%%%%%%%%%%%%%%%%%%%%%%%%%%%%%%%%%%%%%%%%%%%%%%%%%%%%%%%%%%%%%%%%%%%
%%%%%%%%%%%%%%%%%%%%%%%%%%%%%%%%%%%%%%%%%%%%%%%%%%%%%%%%%%%%%%%%%%%%%%%%%%%%

%% \begin{definition}
%% We define the equivalence relation oblivious to flow-sensitive reference
%% implemnetation as:
%% \begin{itemize}
%% \item |LIORefTCB S t &=  LIORefTCB S t'| for any |t|, |t'|.
%% \item |t &=  t'| iff |t = t'|.
%% \item |LIOTCB t &=  LIOTCB t'| iff |t &=  t'|.
%% \item |LabeledTCB l t &= LabeledTCB l t'| iff |t &=  t'| and |l &=  l'|.
%% \item etc.
%% \end{itemize}
%% \end{definition}
%% 
%% \begin{theorem}[Equivalence]
%% For all |t, v| in \lio{} and |t', v'| in \liofs{},
%% and |v &= v'|, it must be that
%% %
%% |conf (lcurr,mI) t  ==> conf (lcurr',...) diverge| $\Leftrightarrow$
%% |conf (lcurr,mI,emptyset) t' ==> conf (lcurr',...) diverge| and 
%% %
%% |conf (lcurr,mI) t  ==> conf (lcurr',...) v| $\Leftrightarrow$
%% |conf (lcurr,mI,emptyset) t' ==> conf (lcurr',...) v'|.
%% \end{theorem}


% Local Variables:
% TeX-master: "main.tex"
% TeX-command-default: "Make"
% End:
