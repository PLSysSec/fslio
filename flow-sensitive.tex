\section{Flow-sensitivity extensions}

Previous versions of |LIO| support mutable references (LIORefs) with a
fixed label, which is preserved throughout the computation. These
references are known as \emph{flow-insensitive}. In this work, we
extend |LIO| to include \emph{flow-sensitive} references,
i.e. references which can store values with different labels over the
course of the computation. Unlike their flow-insensitive counterparts,
the label associated with such a reference is mutable, and will vary
according to the sensitivity of what is stored in the
reference. Writing a secret value to a public reference in this
context will cause the reference to be relabelled as secret, instead
of halting the program.

Flow-sensitive references allow the enforcement mechanism to accept a
greater number of secure programs. Nevertheless,
Russo~and~Sabelfeld~\citep{?} show that a naive implementation of this
idea in a dynamic enforcement introduces a security leak. We
illustrate this attack in the context of |LIO| in
Figure~\ref{fig:fs-attack}. In this function, we assume that we have a
version of |LIO| where |LIORef|s are flow-sensitive, so the
|writeLIORef| operations in the conditionals can raise the label of
the reference they assign to, if the current label is $H$. The
function takes both a |public| and a |secret| reference as parameters,
which contain either 1 or 0. Then, the function creates a public
reference |temp| with initial value 0, and assigns 1 to |public|.
Note that, if the value of |secret| is 1, the first |writeLIORef| is
executed, but this causes |temp| to be 1, which prevents the second
|writeLIORef| from executing. Thus, |public| is untouched and it will
contain the value of |secret|. If the value of |secret| is 0, the
first |writeLIORef| is not executed, but the second one is, causing
|public| to be assigned the value 0. Since |temp| has not been
assigned in a secret context, its label remains $L$, and so does the
label of |public|. In either case, the final value of |public| matches
the value in |secret|, so one bit of information has been leaked
through this attack, without using the termination channel.
% maybe say that the key part of the attack is the fact that temp
% has different labels in different runs of the progam ???

\begin{figure}[!ht]
\vspace*{-5pt}
\begin{code}
attack public secret = 
  do  temp    <- newLIORef L 0
      writeLIORef public 1
      toLabeled H $ do  x <- readLIORef secret
                        when (x == 1) (writeLIORef temp 1)
      toLabeled H $ do  c <- readLIORef temp
                        when (c /= 1) (writeLIORef public 0)
\end{code}
\caption{Potential flow-sensitivity attack in |LIO|.\label{fig:fs-attack}}
\vspace*{-5pt}
\end{figure}

This problem can be addressed in several ways. Prior work proposed the
\emph{no-sensitive-upgrade} strategy~\citep{?} and the
\emph{permissive-upgrade} strategy~\citep{?}. The no-sensitive-upgrade
check forbids assignment to a public reference in a secret context. In
contrast, the permissive-upgrade check allows the assignment but marks
it as a partial leak; if there is a subsequent conditional branch on
partially leaked data, the program stops. We adapt the
permissive-upgrade idea to our floating-label model, which results in
a slightly more permissive enforcement that allows the conditional
branch.

We extend the formalization of |LIO| to include both flow-insensitive
and flow-sensitive LIORefs. Figure~\ref{fig:fs-exts-syntax} introduces
the syntax for values, terms and types that we require in order to
account for both flow-sensitive and flow-insensitive references.

\begin{figure}[!ht] % syntax
%%-deian-\vspace*{-5pt}
%%-deian-\begin{align*}
%%-deian-\textrm{Value:}   && v    \Coloneqq~ &
%%-deian-                                 \LIORefRTS{l}{t}
%%-deian-                           \ \mid\  \Lb{l}{u}{t} \\
%%-deian-\textrm{Term:}    && t    \Coloneqq~ &
%%-deian-                             \newRef{l}{t}
%%-deian-                           \mid  \writeRef{t}{t} \\
%%-deian-                  && &     \mid  \readRef{t}
%%-deian-                           \mid  \labelOf{t} \\
%%-deian-                  && &     \mid  \upgrade{t}{t} \\
%%-deian-\textrm{Type:}    && \tau \Coloneqq~& 
%%-deian-                          \LIORef{\ell}{\tau} \\
%%-deian-\end{align*}
\centering
%format Values  = "\mathrm{Values}"
%format Terms   = "\mathrm{Terms}"
%format Types   = "\mathrm{Types}"
%format Sensitivity = "\mathrm{Sensitivity}"
\begin{code}
v   ::= cdots  | LIORefTCB s v t
t   ::= cdots  | newRef s t t | writeRef s t t 
               | readRef s t | upgrade t t
tau ::= cdots  | LIORef s tau


Ep   ::= cdots  | newRef s Ep t | writeRef s Ep t 
                | readRef s Ep | upgrade Ep t | upgrade v Ep
\end{code}
\caption{Formal syntax for values, terms, and types,
where |s  ::=  S  divisor I|.\label{fig:fs-exts-syntax}}
\vspace*{-5pt}
\end{figure} 

We also define the upgrade operator $\uparrow$ as follows:
\begin{code}
  upgradeM m l' r = m { a mapsto (upgradeL (m(a)) l') | a element dom(m) or r}
\end{code}
\begin{code}
  upgradeL ((LabeledTCB l t)) l' = LabeledTCB l'' t where denot(l'' = l lub l')
\end{code}
%LabeledTCB (l' lub l) e 
%\[
%\begin{array}{ll}
%\RTS\uparrow^{l'} r = \RTS \oplus \{ &x\mapsto \Lb{l'}{u}{v} \mid x \in \mbox{dom}(\RTS)\cap r, \\
% &\Lb{l}{u}{v} = \RTS(x), l\flows l'\; \}
%\end{array}
%\]

%%-deian-The expression $f\oplus g$ overrides every binding $x\mapsto y$ in $f$, where
%%-deian-$x\in \mbox{dom}(g)$, with the binding $x\mapsto g(x)$, leaving the rest intact.
%%-deian-
%%-deian-The store $\RTS\uparrow^{l'} r$ has the same
%%-deian-bindings as $\RTS$, where each variable in $r$ has had its label
%%-deian-upgraded to $l'$, provided that the original label was below $l'$.
%%-deian-
%%-deian-% Figure~\ref{fig:fs-exts-semantics} introduces the semantics for LIORef
%%-deian-% operations. Rule $\textsc{NewRef}$ creates a new LIORef with a given
%%-deian-% initial label and value, adding it to the mutable reference store
%%-deian-% $\RTS$. The following rules come in two versions, one for
%%-deian-% flow-insensitive variables and another one for flow-sensitive ones,
%%-deian-% marked as $\textsc{FI}$ and $\textsc{FS}$ respectively. Rules
%%-deian-% $\textsc{ReadRef-FI}$, $\textsc{WriteRef-FI}$ and
%%-deian-% $\textsc{LabelOf-FI}$ are exactly like in previous versions of the
%%-deian-% library. In rule $\textsc{ReadRef-FS}$, in addition to the usual
%%-deian-% tainting of the current label $\env.\lbl$, we have a taint on
%%-deian-% $\env.\fel$, which is used to keep track of the taint caused by
%%-deian-% flow-sensitive variables within |toLabeled| blocks. Rule
%%-deian-% $\textsc{WriteRef-FS}$ causes the label in the flow-sensitive
%%-deian-% reference to be replaced by the current label, allowing operations
%%-deian-% that would otherwise trigger a security violation exception on a
%%-deian-% flow-insensitive variable. Rule $\textsc{LabelOf-FS}$ works by
%%-deian-% tainting the current label with the label of the reference being
%%-deian-% inspected, before returning this label.
%%-deian-
%%-deian-%
%%-deian-% ReadRef and LabelOf rules that propagate the taint outside toLabeled
%%-deian-%
%%-deian-% \inferrule{l\flows\env.\clr \\ \env' = \env[\lbl\mapsto \env.\lbl \lub l] \\ \env'' = \neg u ? \env'[\fel\mapsto \env'.\lbl] : \env' \\  \Lb{l}{u}{v} = \RTS(x)}
%%-deian-% {\defRTS{E[\readRefFS{(\LIORefRTS{l}{x})}]} \lto \wEnvRTS{\env''}{\RTS}{E[\returnLIO{v}]}}[\textsc{ReadRef-FS}]
%%-deian-% \and
%%-deian-% \inferrule{l \flows \env.\clr \and \Lb{l}{u}{\_} = \RTS(x) \\ \env' = \env[\lbl\mapsto\env.\lbl \lub l] \and \env''= \neg u ? \env'[\fel\mapsto\env'.\lbl] : \env'}
%%-deian-% {\defRTS{E[\labelOfFS{(\LIORefRTS{l}{x})}]} \lto \wEnvRTS{\env''}{\RTS}{r}{E[\returnLIO{l}]}}[\textsc{LabelOf-FS}]
%%-deian-
%%-deian-
%%-deian-\begin{figure*}[!ht] % semantics
%%-deian-\vspace*{-5pt}
%%-deian-\small
%%-deian-\begin{mathpar}
%%-deian-\inferrule{\env.\lbl \flows l \flows \env.\clr \and \RTS' = \RTS[ x \mapsto \Lb{l}{\bot}{v} ]\and x\mbox{ fresh}}
%%-deian-{\defRTS{E[\newRef{l}{t}]} \lto \wEnvRTS{\env}{\RTS'}{x\vtl r}{E[\returnLIO{(\LIORefRTS{l}{x})}]}}[\textsc{NewRef}]
%%-deian-%\and
%%-deian-% \inferrule{l\flows\env.\clr \\ \env' = \env[\lbl\mapsto \env.\lbl \lub l] \and  \Lb{l}{u}{v} = \RTS(x)}
%%-deian-% {\defRTS{E[\readRefFI{(\LIORefRTS{l}{x})}]} \lto \wEnvRTS{\env'}{\RTS}{E[\returnLIO{v}]}}[\textsc{ReadRef-FI}]
%%-deian-\and
%%-deian-\inferrule{l\flows\env.\clr \\ \env' = \env[\lbl\mapsto \env.\lbl \lub l] \and \RTS'=\RTS\uparrow^{\env'.\lbl} r \\  \Lb{l}{u}{v} = \RTS(x)}
%%-deian-{\defRTS{E[\readRef{(\LIORefRTS{l}{x})}]} \lto \wEnvRTS{\env'}{\RTS'}{r}{E[\returnLIO{v}]}}[\textsc{ReadRef}]
%%-deian-\and
%%-deian-\inferrule{\env.\lbl \flows l \flows \env.\clr \and \RTS' = \RTS[ x \mapsto \Lb{l}{\bot}{v} ] }
%%-deian-{\defRTS{E[\writeRefFI{(\LIORefRTS{l}{x})}{v}]} \lto \wEnvRTS{\env}{\RTS}{r}{E[\returnLIO{()}]}}[\textsc{WriteRef-FI}]
%%-deian-\and
%%-deian-\inferrule{l \flows \env.\lbl \and \RTS' = \RTS[ x \mapsto \Lb{\env.\lbl}{\bot}{v} ] \and x \in r}
%%-deian-{\defRTS{E[\writeRefFS{(\LIORefRTS{l}{x})}{v}]} \lto \wEnvRTS{\env}{\RTS}{r}{E[\returnLIO{()}]}}[\textsc{WriteRef-FS}]
%%-deian-\and
%%-deian-% \inferrule{l \flows \env.\lbl \\ \Lb{l}{u}{\_} = \RTS(x) \and \RTS' = \RTS[ x \mapsto \Lb{\env.\lbl}{u}{v} ]}
%%-deian-% {\defRTS{E[\writeRefFSI{(\LIORefRTS{l}{x})}{v}]} \lto \wEnvRTS{\env}{\RTS}{r}{E[\returnLIO{()}]}}[\textsc{WriteRef-FSI}]
%%-deian-% \and
%%-deian-\inferrule{\env.\lbl\flows l \and l\flows l' \and \RTS'=\RTS\uparrow^{l'} \{x\}}
%%-deian-{\defRTS{E[\upgrade{(\LIORef{l}{x})}{l'}]} \lto \wEnvRTS{\env}{\RTS'}{r}{E[\returnLIO{()}]}}[\textsc{Upgrade}]
%%-deian-\and
%%-deian-\inferrule{}
%%-deian-{\defRTS{E[\labelOfFI{(\LIORefRTS{l}{x})}]} \lto \wEnvRTS{\env}{\RTS}{r}{E[\returnLIO{l}]}}[\textsc{LabelOf-FI}]
%%-deian-\and
%%-deian-\inferrule{l \flows \env.\clr \and \Lb{l}{u}{\_} = \RTS(x) \\ \env' = \env[\lbl\mapsto\env.\lbl \lub l] \and \RTS'=\RTS\uparrow^{\env'.\lbl} r }
%%-deian-{\defRTS{E[\labelOfFS{(\LIORefRTS{l}{x})}]} \lto \wEnvRTS{\env'}{\RTS'}{r}{E[\returnLIO{l}]}}[\textsc{LabelOf-FS}]
%%-deian-\end{mathpar}
%%-deian-\caption{Semantics for references.\label{fig:fs-exts-semantics}}
%%-deian-\vspace*{-5pt}
%%-deian-\end{figure*}
%%-deian-
%%-deian-%
%%-deian-% ToLabeled rule that propagates FS taints
%%-deian-%
%%-deian-% \inferrule{\env.\lbl \flows l \flows \env.\clr \\ \eEnvRTS{\env}{\RTS}{r'}{t}\lto^*\wEnvRTS{\env'}{r''}{\RTS'}{\lioValp{t'}} \\ \env'' = \env[\lbl\mapsto\env.\lbl \lub \env'.\fel, \fel\mapsto\env'.\fel]}
%%-deian-% {\defRTS{E[\toLabeled{l}{t}]} \lto \\ \wEnvRTS{\env''}{\RTS'}{r}{E[\returnLIO{(\Lb{l}{\bot}{t'})}]}}[\textsc{toLabeled}]
%%-deian-
%%-deian-
%%-deian-\begin{figure}[!ht] % semantics
%%-deian-\vspace*{-5pt}
%%-deian-\small
%%-deian-\begin{mathpar}
%%-deian-\inferrule{\env.\lbl \flows l \flows \env.\clr \and r' \subseteq r \\ \wEnvRTS{\env}{\RTS}{r'}{t}\lto^*\wEnvRTS{\env'}{r''}{\RTS'}{\lioValp{t'}} }
%%-deian-{\defRTS{E[\toLabeled{l\ r'}{t}]} \lto \\ \wEnvRTS{\env}{\RTS'}{r\cup r''}{E[\returnLIO{(\Lb{l}{\bot}{t'})}]}}[\textsc{toLabeled}]
%%-deian-\end{mathpar}
%%-deian-\caption{Semantics for |toLabeled|.\label{fig:toLabeled-semantics}}
%%-deian-\vspace*{-5pt}
%%-deian-\end{figure}
%%-deian-
%%-deian-\begin{figure}
%%-deian-\vspace*{-5pt}
%%-deian-\begin{code}
%%-deian-toLabeledWith :: Label l =>
%%-deian-      l -> a -> (LIORef FS l a -> LIO l b)
%%-deian-  ->  LIO l (Labeled FS l b)
%%-deian-toLabeledWith l def f =
%%-deian-    do  y    <-  newLIORef l def
%%-deian-        upgrade y l -- set the upgraded flag
%%-deian-        ret  <-  toLabeled [y] (f y)
%%-deian-        destroyLIORef y
%%-deian-        return ret
%%-deian-\end{code}
%%-deian-\caption{Definition of |toLabeledWith|.\label{fig:toLabeledWith-semantics}}
%%-deian-\vspace*{-5pt}
%%-deian-\end{figure}
%%-deian-
%%-deian-% Figure~\ref{fig:toLabeled-semantics} shows the reduction rule for
%%-deian-% |toLabeled| operations. In this rule, we see that the $\env.\fel$
%%-deian-% label is used to taint the current label after the |toLabeled|
%%-deian-% operation is complete. This allows any partially-leaked information
%%-deian-% (due to flow-sensitive references) to be taken into account in future
%%-deian-% label checks.
