\section{Flow-sensitivity extensions}

Previous versions of |LIO| support mutable references (LIORefs) with a
fixed label, which is preserved throughout the computation. These
references are known as \emph{flow-insensitive}. In this work, we
extend |LIO| to include \emph{flow-sensitive} references,
i.e. references which can store values with different labels over the
course of the computation. Unlike their flow-insensitive counterparts,
the label associated with such a reference is mutable, and will vary
according to the sensitivity of what is stored in the
reference. Writing a secret value to a public reference in this
context will cause the reference to be relabelled as secret, instead
of halting the program.

Flow-sensitive references allow the enforcement mechanism to accept a
greater number of secure programs. Nevertheless,
Russo~and~Sabelfeld~\citep{?} show that a naive implementation of this
idea in a dynamic enforcement introduces a security leak. We
illustrate this attack in the context of |LIO| in
Figure~\ref{fig:fs-attack}. In this function, we assume that we have a
version of |LIO| where |LIORef|s are flow-sensitive, so the
|writeLIORef| operations in the conditionals can raise the label of
the reference they assign to, if the current label is $H$. The
function takes both a |public| and a |secret| reference as input,
which contain either 1 or 0. Then, the function creates a public
reference |temp| with initial value 0, and assigns 1 to |public|.
Note that, if the value of |secret| is 1, the first |writeLIORef| is
executed, but this causes |temp| to be 1, which prevents the second
|writeLIORef| from executing. Thus, |public| is untouched and it will
contain the value of |secret|. If the value of |secret| is 0, the
first |writeLIORef| is not executed, but the second one is, causing
|public| to be assigned the value 0. Since |temp| has not been
assigned in a secret context, its label remains $L$, and so does the
label of |public|. In either case, the final value of |public| matches
the value in |secret|, so one bit of information has been leaked
through this attack, without using the termination channel.
% maybe say that the key part of the attack is the fact that temp
% has different labels in different runs of the progam ???

\begin{figure}[!ht]
\vspace*{-5pt}
\begin{code}
attack public secret = 
  do  temp    <- newLIORef L 0
      writeLIORef public 1
      toLabeled H $ do  x <- readLIORef secret
                        when (x == 1) (writeLIORef temp 1)
      toLabeled H $ do  c <- readLIORef temp
                        when (c /= 1) (writeLIORef public 0)
\end{code}
\caption{Potential flow-sensitivity attack in |LIO|.\label{fig:fs-attack}}
\vspace*{-5pt}
\end{figure}

This problem can be addressed in several ways. Prior work proposed the
\emph{no-sensitive-upgrade} strategy~\citep{?} and the
\emph{permissive-upgrade} strategy~\citep{?}. The no-sensitive-upgrade
check forbids assignment to a public reference in a secret context. In
contrast, the permissive-upgrade check allows the assignment but marks
it as a partial leak; if there is a subsequent conditional branch on
partially leaked data, the program stops. We adapt the
permissive-upgrade idea to our floating-label model, which results in
a slightly more permissive enforcement that allows the conditional
branch. We believe that our enforcement is just as permissive as %% I don't like this sentence
permissive-upgrades extended with dynamic insertion of privatization
operations~\citep{?}.

We extend the formalization of |LIO| to include both flow-insensitive
and flow-sensitive LIORefs. Figure~\ref{fig:fs-exts-syntax} introduces
the syntax for values, terms and types that we require in order to
account for both flow-sensitive and flow-insensitive references.

\begin{figure}[!ht] % syntax
\vspace*{-5pt}
\begin{align*}
\textrm{Value:}   && v    \Coloneqq~ &
                                 \LIORefRTS{l}{t}
                           \ \mid\  \Lb{l}{t} \\
\textrm{Term:}    && t    \Coloneqq~ &
                             \newRef{l}{t}
                           \mid  \writeRef{t}{v} \\
                  && &     \mid  \readRef{t}
                           \mid  \labelOf{t} \\
\textrm{Type:}    && \tau \Coloneqq~& 
                          \LIORef{\ell}{\tau} \\
\end{align*}
\caption{Formal syntax for values, terms, and types.\label{fig:fs-exts-syntax}}
\vspace*{-5pt}
\end{figure} 

Figure~\ref{fig:fs-exts-semantics} introduces the semantics for LIORef
operations. Rule $\textsc{NewRef}$ creates a new LIORef with a given
initial label and value, adding it to the mutable reference store
$\RTS$. The following rules come in two versions, one for
flow-insensitive variables and another one for flow-sensitive ones,
marked as $\textsc{FI}$ and $\textsc{FS}$ respectively. Rules
$\textsc{ReadRef-FI}$, $\textsc{WriteRef-FI}$ and
$\textsc{LabelOf-FI}$ are exactly like in previous versions of the
library. In rule $\textsc{ReadRef-FS}$, in addition to the usual
tainting of the current label $\env.\lbl$, we have a taint on
$\env.\fel$, which is used to keep track of the taint caused by
flow-sensitive variables within |toLabeled| blocks. Rule
$\textsc{WriteRef-FS}$ causes the label in the flow-sensitive
reference to be replaced by the current label, allowing operations
that would otherwise trigger a security violation exception on a
flow-insensitive variable. Rule $\textsc{LabelOf-FS}$ works by
tainting the current label with the label of the reference being
inspected, before returning this label.

\begin{figure}[!ht] % semantics
\vspace*{-5pt}
\small
\begin{mathpar}
\inferrule{\env.\lbl \flows l \flows \env.\clr \and \RTS' = \RTS[ x \mapsto \Lb{l}{v} ]\and x\mbox{ fresh}}
{\defRTS{E[\newRef{l}{t}]} \lto \\ \wEnvRTS{\env}{\RTS'}{E[\returnLIO{(\LIORefRTS{l}{x})}]}}[\textsc{NewRef}]
\and
\inferrule{\env' = \env[\lbl\mapsto \env.\lbl \lub l] \and  \Lb{l}{v} = \RTS(x)}
{\defRTS{E[\readRefFI{(\LIORefRTS{l}{x})}]} \lto \\ \wEnvRTS{\env'}{\RTS}{E[\returnLIO{v}]}}[\textsc{ReadRef-FI}]
\and
\inferrule{\env' = \env[\lbl\mapsto \env.\lbl \lub l] \\ \env' = \env[\fel\mapsto \env.\fel \lub l] \\  \Lb{l}{v} = \RTS(x)}
{\defRTS{E[\readRefFS{(\LIORefRTS{l}{x})}]} \lto \\ \wEnvRTS{\env'}{\RTS}{E[\returnLIO{v}]}}[\textsc{ReadRef-FS}]
\and
\inferrule{\env.\lbl \flows l \flows \env.\clr \and \RTS' = \RTS[ x \mapsto \Lb{l}{v} ] }
{\defRTS{E[\writeRefFI{(\LIORefRTS{l}{x})}{v}]} \lto \\ \wEnvRTS{\env}{\RTS}{E[\returnLIO{()}]}}[\textsc{WriteRef-FI}]
\and
\inferrule{l \flows \env.\lbl \and \RTS' = \RTS[ x \mapsto \Lb{\env.\lbl}{v} ]}
{\defRTS{E[\writeRefFS{(\LIORefRTS{l}{x})}{v}]} \lto \\ \wEnvRTS{\env}{\RTS}{E[\returnLIO{()}]}}[\textsc{WriteRef-FS}]
\and
\inferrule{}
{\defRTS{E[\labelOfFI{(\LIORefRTS{l}{x})}]} \lto \\ \wEnvRTS{\env}{\RTS}{E[\returnLIO{l}]}}[\textsc{LabelOf-FI}]
\and
\inferrule{l \flows \env.\clr \and \env' = \env[\lbl\mapsto\env.\lbl \lub l]}
{\defRTS{E[\labelOfFS{(\LIORefRTS{l}{x})}]} \lto \\ \wEnvRTS{\env'}{\RTS}{E[\returnLIO{l}]}}[\textsc{LabelOf-FS}]
\end{mathpar}
\caption{Semantics for references.\label{fig:fs-exts-semantics}}
\vspace*{-5pt}
\end{figure}

\begin{figure}[!ht] % semantics
\vspace*{-5pt}
\small
\begin{mathpar}
\inferrule{\env.\lbl \flows l \flows \env.\clr \\ \defRTS{t}\lto^*\wEnvRTS{\env'}{\RTS'}{\lioValp{t'}} \\ \env'' = \env[\lbl\mapsto\env.\lbl \lub \env'.\fel]}
{\defRTS{E[\toLabeled{l}{t}]} \lto \\ \wEnvRTS{\env''}{\RTS'}{E[\returnLIO{(\Lb{l}{t'})}]}}[\textsc{toLabeled}]
\end{mathpar}
\caption{Semantics for |toLabeled|.\label{fig:toLabeled-semantics}}
\vspace*{-5pt}
\end{figure}

Figure~\ref{fig:toLabeled-semantics} shows the reduction rule for
|toLabeled| operations. In this rule, we see that the $\env.\fel$
label is used to taint the current label after the |toLabeled|
operation is complete. This allows any partially-leaked information
(due to flow-sensitive references) to be taken into account in future
label checks.
