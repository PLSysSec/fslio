\section{Flow-sensitivity extensions}
\label{sec:flow-sensitive}

%\concept{a more-LIO focused example for why FS}
Unfortunately, the flow-insensitive references described in the previous section
are somewhat inflexible.
%
Consider, for example, an application that uses a reference as a log.
%
Since the log may contain sensitive information, it is important that the
reference be labeled.
%
Equally important is to be able to read the log at any point in the program to,
for instance, save it to a file.
%
Although labeling the reference with the top element in the
security lattice (i.e., $\top$) would always allow writes to the log, and |toLabeled| can be
used to write the log to a file, this is unsatisfactory: it assumes the
existence of a top element, which in practical IFC systems such as
Hails~\cite{giffin:hails}, does not exist.
%
More importantly, it almost always over-approximates the sensitivity of the
log. For instance, suppose that the computation never reads sensitive data, yet wishes to
read the log content to, e.g., format it, then it needs to taint itself to $\top$!
%
It is clear that even in the case of simple scenarios, having references with
mutable labels that vary according to the sensitive of what is stored in the
reference is useful.

%\concept{naive approach doesn't work}
% Ale: I don't like implicit since it might be associated to implicit flows
However, naively implementing flow-sensitive references can effectively
introduce label changes as a covert channel. 
%similar to other language-based system to LIO.
%
Suppose that we allow for the label of a reference to simply be raised to the
current label at the time a |writeRef space| is performed.
%
So, for example, if the label of our log reference is |low| and the computation
has read sensitive data such that the current label is |H|, now writing to the
log raises the label of the reference to |H|.
%
As identified by other language-based IFC systems
(e.g. \cite{Russo:2010,Austin:Flanagan:PLAS09,Austin:Flanagan:PLAS10}, this
approach is not safe; consider the code in Figure~\ref{fig:fs-attack}.
%While this approach was clearly shown to be unsafe, due to implicit-flows, in
%language-based systems~\tocite{}, LIO eliminates the implicit-flows problem by
%construction and thus raises the question: \emph{is this approach safe in LIO?}
%
%Unfortunately, the answer to this question is \emph{no} 
%
\begin{figure}[t]
\small
\begin{code}
leakRef :: LIORefTCB space Bool -> LIO Bool
leakRef href = do
  tmp   <- newRef space low ()
  toLabeled high $ do  h <- readRef space href
                       when h (writeRef space tmp ())
  return (labelOfR space tmp == high)
\end{code}
\cut{$}
\caption{Attack in LIO with naive flow-sensitive reference extension. We ommit
subscripts on reference terminals for clarity.  \label{fig:fs-attack}}
\end{figure}

%\concept{can leak into ref label}
%\noindent
%
In this attack, |leakRef| can be used to leak the value stored in a |high| reference
|href|, while keeping the current label |low|, by leveraging a new
covert channel: the label of references.
%
Concretely, in |leakRef| we create a public reference |tmp| and withing the
|toLabeled| block, which is used to ensure that the current label remains
|low|, the label of this reference is changed to |H| if the secret stored in
|href| is |true|, and left intact (|L|) if the secret is |false|.
%
Then, simply inspecting the label of the reference reveals the value stored in
|href|!
%
We note that the use of |labelOfR space| is not fundamental to this attack; in
Appendix~\ref{sec:app:fs-attack2} we show an alternative version of |leakRef|
which does not rely on label inspection and follows the structure of the attack
given in Section~\ref{sec:intro}.

%\concept{KEY INSIGHT:}
Fundamentally, the label protecting the \emph{label} of an object (e.g.,
labeled values and references) is the current label |lcurr| at the time of
creation.
%
Hence, to modify the label of the object within some context (e.g., |toLabeled|
block) wherein the current label is |lcurr'|, \emph{it must be the case that} |lcurr'
canFlowTo lcurr|, i.e., we must be able to write data at sensitivity level
|lcurr'| into an entity (the label) labeled |lcurr|.
%
This restriction is especially important if we can restore the current label
from |lcurr'|, where |lcurr canFlowToStrict lcurr'|,  
to |lcurr| and subsequently observe the leak within the program itself.
%
In the case where the label of the object is immutable, as is the case for
labeled values and flow-insensitive references, this is not a concern: even if
the current label is raised to |lcurr'| and then restored to |lcurr|, we do not
learn any information more sensitive than |lcurr| (the label of the label at the
time of creation) by inspecting the label of the reference: the label has not
changed!
%

\begin{figure} % semantics
\small
\begin{code}
v    ::= cdots  | LIORefTCB S t
t    ::= cdots  | upgradeR t t | diverge
Ep   ::= cdots  | upgradeR Ep t | upgradeR v Ep
\end{code}

\begin{mathpar}
\inferrule[newRef-|S|]
{ |c = (lcurr, mI, mS)|\\
  |lcurr canFlowTo l|\\
  |fresh a|\\
  |mS' = mS[a mapsto LabeledTCB lcurr (LabeledTCB l t)]|\\
  |c' = (lcurr, mI, mS')|
}
{ |conf c (E[newRef S l t]) --> conf c' (E[return (LIORefTCB S a)])| }

\and

\inferrule[readRef-|S|]
{ |c = (lcurr, mI, mS)| \\
  |mS(a) = LabeledTCB l (LabeledTCB l' t)|\\
  |l'' = l lub l'|
}
{ |conf c (E[readRef S (LIORefTCB S a)]) --> conf c (E[unlabel (LabeledTCB l'' t)])| }

\and

\inferrule[writeRef-|S|]
{ |c = (lcurr, mI mS)|\\
  |mS(a) = LabeledTCB l (LabeledTCB l' t)| \\
  |lcurr canFlowTo (l lub l')|\\
  |mS' = mS[a mapsto LabeledTCB l (LabeledTCB l' t)]|\\
  |c' = (lcurr, mI, mS')|
}
{ |conf c (E[writeRef S (LIORefTCB S a) t]) --> conf c' (E[return ()])| }

\and

\inferrule[writeRef-|S|-fail]
{ |c = (lcurr, mI mS)|\\
  |mS(a) = LabeledTCB l (LabeledTCB l' t)| \\
  |lcurr cantFlowTo (l lub l')|
}
{ |conf c (E[writeRef S (LIORefTCB S a) t]) --> conf c' (E[unlabel (LabeledTCB l diverge)])| }

\and

\inferrule[labelOf-|S|]
{ |c = (lcurr, mI, mS)|\\
  |mS(a) = LabeledTCB l (LabeledTCB l' t)|
}
{ |conf c (E[labelOfR S (LIORefTCB S a)]) --> conf c (E[unlabel (LabeledTCB l l')])|}

\and

\inferrule[upgradeRef]
{ |c = (lcurr, mI, mS)|\\
  |mS(a) = LabeledTCB l' v|\\
  |lcurr canFlowTo l' ~> true| \\
  |conf c (upgrade v l) ==> conf c (LIOTCB v')|\\
  |mS' = mS [ a mapsto LabeledTCB l' v']|\\
  |c' = (lcurr, mI, mS')|
}
{ |conf c (E[upgradeR (LIORefTCB S a) l]) --> conf c' (E[return ()])| }
\end{mathpar}
\caption{\liofs{}: \lio{} with flow-sensitive references.\label{fig:sos:fs}}
\end{figure}


%\concept{\liofs{}}
Thus, to extend LIO with flow-sensitive reference, we must account for the label
on the label of the reference at the time of creation, |lcurr|.
%
(This label is, however, immutable.)
%
In turn, when changing the label of the reference, we must ensure that no data
from the context at the time of the change, whose label is |lcurr'|, is leaked
into the label of the reference by ensuring that |lcurr' canFlowTo lcurr|,
i.e., we can write data labeled |lcurr'| into the label labeled |lcurr|.
%
Formally, we extend the \lio{} syntax shown in Figure~\ref{fig:sos:refs}
with |s=S| and Figure~\ref{fig:sos:fs};
%
the reduction rules for this extended calculus, called \liofs{}, are also
given in Figure~\ref{fig:sos:fs}.

\concept{\liofs{} semantics}
When creating a flow-sensitive reference, |newRef S l t| creates a labeled
value that guards |t| with label |l| (|LabeledTCB l t|).
%
However, since we wish to allow programmers to modify the label |l| of the
reference, we additionally store the label on |l|, i.e., |lcurr|, by simply
labeling the already-guarded term (|mS' = mS[a mapsto LabeledTCB lcurr
(LabeledTCB l t)]|), as as shown in rule~\ruleref{newRef-|S|} .
%
Primitive |newRef S| returns a |LIORefTCB S a| which simply encapsulates the fresh
reference address where the doubly-labeled term is stored.
%
Different from the constructor |LIORefTCB I|,  the constructor |LIORefTCB S|
does not encapsulate the label of the reference. 
%
This is precisely because the label of a flow-sensitive reference is 
mutable and must be looked up in the store.
%
As given by rule~\ruleref{labelOf-|S|}, |labelOfR S| returns the label of the
reference after raising the current label (with |unlabel|) to account for the
fact that the label of the reference |l'| is a value at sensitivity level |l|,
i.e., we raise the current label to the join of the current label and the label
on the label.
%
The rules for reading and writing to flow-sensitive references are mostly
standard; |readRef S| raise the current label to the label of the reference
(which is above the label on the label) and |writeRef S| ensure that the
current computation can write to the reference (|lcurr canFlowTo (l lub l')|).
%
One peculiarity with |writeRef S| is the case when the current label does not
flow to the join of the reference label, i.e., |lcurr cantFlowTo l lub l'|, and
thus the write should fail.
%
If our semantics simply gets stuck, there is a termination leak which reveals
information about the possible changes on label |l'| (itself protected by |l|).
Observe that those changes could have occurred in a more sensitive context than
|lcurr|. For sequential LIO, this is not a problem since the security guarantees
ignore termination leaks~\cite{stefan:lio}. However, in a concurrent setting,
such leaks are a concern~\cite{stefan:addressing-covert}. Aiming to apply our
flow-sensitive approach for concurrency almost without modifications (see
Section~\ref{sec:conc}), we introduce a failure rule for |writeRef S|
which
%does not reflect the fact
%that the rule relies, and thus observes, the label on the reference |l'| which
%is itself protected by |l|; hence, we provide an explicit rule for this failure
%case that 
first raises the current label (through |unlabel|) and then
divergence (we do not provide a reduction rule for |divergence|).
%
The effects of this rule in a sequential settings is analogous to getting
stuck\footnote{This is not true if we consider exceptions, a feature 
out of the scope of this paper}. 
Additionally, we note that |writeRef S| does not modify the label of the reference.
%
This is, in part, because we wish to keep the difference between
flow-insensitive and flow-sensitive references as close as possible.
%
Instead, we provide |upgradeR| precisely for this purpose; this primitive is
used to raise the label of the reference.
%
Importantly it ensures that the current computation can modify the label by
checking that the current label flows to the label on the label (|lcurr
canFlowTo l'|).




\subsection{Automatic upgrades}
\label{sec:flow-sensitive:auto}

\begin{figure} % auto-fs
\small
\begin{code}
v    ::= cdots  | set (v, ...) | nil
t    ::= cdots  | set (t, ...) | withRefs t t
tau  ::= cdots  | set (tau, ...)
Ep   ::= cdots  | set (Ep, t, ...) | set(v, Ep, t, ...) | withRefs Ep t 

addrs (nil)                                       @= emptyset
addrs (set(LIORefTCB S a1, LIORefTCB S a2, ...))  @= {a1, a2, ...}
\end{code}
\begin{mathpar}
\inferrule[withRefs]
{|c = (lcurr, mI, mS)|\\
 |mS' = {a mapsto mS(a) inlinesep a element dom mS and (addrs v) }|\\
 |conf (lcurr, mI, mS') t ==> conf (lcurr', mI', mS'') (LIOTCB t')|\\
 |c''' = (lcurr', mI', mS'' merge mS)|
}
{|conf c (E[withRefs v t]) --> conf c'' (E[return t']|}
\and
\inferrule[upgradeStore]
{|c = (lcurr, mI, mS)|\\
 |mS = {a1 mapsto v1, ..., a_n mapsto v_n}|\\
 |ti = upgradeR (LIORefTCB S a_i) l, i = 1, ..., n|
}
{|conf c (E[upgradeM l]) --> conf c (E[t1 >> ... >> t_n])|}
\and
\inferrule[auto-upgrade]
{ |c = (lcurr, mI, mS)|\\
  |conf c t --> conf c' t'|\\
  |c' = (lcurr', mI', mS')|\\
  |t'' = if lcurr == lcurr' then t' else upgradeM lcurr' >> t'|
}
{
|conf c t --> conf c' t''|
}
\end{mathpar}
\caption{\lioafs{}: Extending \liofs{} with auto-upgrades.\label{fig:sos:afs}}
\end{figure}

\concept{upgrade is burdensome}
We can use \liofs{} to implement various applications that rely on
flow-sensitive references, even those that rely on policies such as the
popular no-sensitive upgrades of Austin and Flanagan~\cite{Austin:Flanagan:PLAS09}.
%
(In Section~\ref{sec:related}, we describe the encoding of a policy that is
similar to, but more permissive than, no-sensitive upgrades.)
%
Using \liofs{} we can also safely implement our logging application using a
flow-sensitive reference.
%
Unfortunately, our system (and other like it) requires that we insert
|upgrade|s before we raise the current label, if we wish to write to the
reference in the more-sensitive context.

\concept{auto upgrade}
Inspired by~\citep{Hedin13}, we thus provide an extension to \liofs{} that can
be used to automatically upgrade references.
%
This extension, called \lioafs{}, is given in Figure~\ref{fig:sos:afs}.
%
As given by rules \ruleref{auto-upgrade}, whenever the current label is about
to be raised, we first upgrade all the references (with |upgradeM|) in the
store and then raise the current label.\footnote{
We encode the auto-upgrade policy as in rule \ruleref{auto-upgrade} for
brevity, despite the introduction of non-determinism in the semantics.
%
Importantly, and as done in our implementation, the store upgrade is
individually added to the functions that actually raise the current label
(e.g., |unlabel|).
}
%
Intuitively, this ensures that as the computation progresses it does not
``lose'' write access to its references.
%
Continuing our logging example, with auto-upgrades the reference used to the log
never needs to be explicitly upgraded and can truly present an interface of a
logger!

\concept{store creep}
Recall, however, that |toLabeled| provides a means for temporarily raising the
current label.
%
This allows us to write programs without the label creep.
%
Unfortunately, when raising the current label within a |toLabeled| block, the
upgrades of the flow-sensitive references remain even after the current label
is restored.
%
Thus, when reading from any flow-sensitive reference after the |toLabeled|
block will raise the current label to, at least, the current label at the end
of the |toLabeled| block (since all references are upgraded).
%
This can be used to carry out a ``poison pill'' attack~\tocite{breeze}, wherein
the (usually untrusted) computation executing within the |toLabeled| block will
render the outer computation useless by imposing label creep.

\concept{withRefs}
Hence, our \lioafs{} calculus further extends \liofs{} with |withRefs v t| which
takes a bag |v| of references and a computation |t|, and executes |t| in a
configuration where the flow-sensitive reference store only contains the subset
of references |v|.
%
A bag is either empty |nil|, or a number of references |set (v,..)| (of
potentially different types).
%
Rule \ruleref{withRefs} precisely defines the semantics of the new terminal,
where |addrs| converts a bag of references to a set of their corresponding
addresses, and |merge| is used to merge the stores, giving preferences to the
left-hand-side store.
%
The function is mostly standard and provides a programming pattern similar to
that already present in some mainstream languages (e.g., C++'s \Red{ASK DM}),
we do not discuss it further.
%
We lastly remark that to address the poison pill attack, we can now simply wrap
|toLabeled| with |withRefs|, restricting the (untrusted) code from upgrading
arbitrary references.


%%%%%%%%%%%%%%%%%%%%%%%%%%%%%%%%%%%%%%%%%%%%%%%%%%%%%%%%%%%%%%%%%%%%%%%%%%%%
%%%%%%%%%%%%%%%%%%%%%%%%%%%%%%%%%%%%%%%%%%%%%%%%%%%%%%%%%%%%%%%%%%%%%%%%%%%%

%% \begin{definition}
%% We define the equivalence relation oblivious to flow-sensitive reference
%% implemnetation as:
%% \begin{itemize}
%% \item |LIORefTCB S t &=  LIORefTCB S t'| for any |t|, |t'|.
%% \item |t &=  t'| iff |t = t'|.
%% \item |LIOTCB t &=  LIOTCB t'| iff |t &=  t'|.
%% \item |LabeledTCB l t &= LabeledTCB l t'| iff |t &=  t'| and |l &=  l'|.
%% \item etc.
%% \end{itemize}
%% \end{definition}
%% 
%% \begin{theorem}[Equivalence]
%% For all |t, v| in \lio{} and |t', v'| in \liofs{},
%% and |v &= v'|, it must be that
%% %
%% |conf (lcurr,mI) t  ==> conf (lcurr',...) diverge| $\Leftrightarrow$
%% |conf (lcurr,mI,emptyset) t' ==> conf (lcurr',...) diverge| and 
%% %
%% |conf (lcurr,mI) t  ==> conf (lcurr',...) v| $\Leftrightarrow$
%% |conf (lcurr,mI,emptyset) t' ==> conf (lcurr',...) v'|.
%% \end{theorem}


% Local Variables:
% TeX-master: "main.tex"
% TeX-command-default: "Make"
% End:
