\section{Brief introduction to LIO}
\label{sec:background}

LIO is a language-level IFC system, implemented as a Haskell
library.
%
Specifically, LIO provides programmers with a new monad, called |LIO|,
that is similar to---and is intended to be used in place of---the
standard Haskell |IO| monad, but additionally monitors and controls
the flow of information.
%
The LIO library provides various practical features (e.g., mutable
references and threads) as combinators in this monad.

\Red{TODO: describe LIO monad and current label}

%
Below we describe the semantics of LIO as an extended simply-typed,
call-by-name $\lambda$-calculus, called \lio, parametric in the label
type |lattice|.

The formal syntax of the core \lio{} calculus is given in
Fig.~\ref{fig:language:base}.
%
Syntactic categories |v|, |t|, and |tau| represent values, terms, and
types, respectively.
%
Values include standard primitives (Booleans, unit,
and $\lambda$-abstractions) and terminals corresponding to labels
(|lattice|) and monadic values (|LIOTCB tau|).\footnote{
We restrict our formalization to computations implemented in the |LIO|
monad and only consider Haskell features relevant to IFC, similar to
the presentation of LIO in~\tocite{jfptechreport}.
}
%
Terms constitute standard constructs (values, variables |x|, function
application, the |fix| operator, and conditionals), terminals
corresponding to label operations (|t oplus t|), standard monadic
operators (|return t| and |t >>= t|), and |getLabel|, describe below.
%
Types consist of Booleans, unit, function types, labels, and |LIO|
computations; since the \lio{} type system is standard, we do not
discuss it further.

\begin{figure}
\centering
%format Values  = "\mathrm{Values}"
%format Terms   = "\mathrm{Terms}"
%format Types   = "\mathrm{Types}"
%format LOps    = "\mathrm{Ops}_"lattice
\begin{code}
Values  v      ::=  true | false | () | \x.t | lattice | LIOTCB t
Terms   t      ::=  v | x | t t | fix t | if t then t else t
                 |  t oplus t | return t | t >>= t | getLabel
Types   tau    ::=  Bool | () | tau -> tau | lattice | LIO tau  
LOps    oplus  ::=  canFlowTo | lub | glb
\end{code}
\caption{Base \lio{} syntactic categories.
\label{fig:language:base}}
\end{figure}

An LIO program is a \emph{configuration} of the form
|conf lcurr m t|, where |lcurr| is the current label, |m|
is the memory store, and |t| is
the term under evaluation.
%
(We note that for a program to be valid | :- t : LIO tau|.)



\begin{code}
Ep  ::= Ep t | fix Ep | if Ep then t else t 
      | Ep oplus t | v oplus Ep
E  ::= []| Ep | E >>= t 
\end{code}

\Red{
Maybe useful somewhere else:
The design constraints imposed by the library approach has led to a
novel point in the design space of language-level IFC systems---when
compared to more traditional IFC systems---that incorporates ideas
from previous IFC operating systems.
}
