\section{Related work}

%% Flow-sensitive monitors 
Hunt and Sands~\citep{Hunt:2006} show the equivalence (modulo code
transformation) between flow-sensitive and flow-insensitive type-systems. In a
dynamic setting, Russo and Sabelfeld~\citep{Russo:2010} formally pin down the
menace of mutuable label for purely dynamic monitors. They prove that monitors
require static analysis in order to be more permissive than traditional
flow-sensitive type-systems. Targeting purely dynamic monitors, Austin and
Flanagan~\citep{Austin:Flanagan:PLAS10} provide the label changes policies
\emph{\nsu} and \emph{\pu}, where the latter is provably more permissive
(i.e. it rejects less suspicious programs). The discipline {\nsu} stops the
execution on any attempt to change the label of a public variable inside a
secret context. In contrast, {\pu} allows such changes but marking the altered
variables so that the program cannot branch on them. Unfortunately, none of the
mentioned work so far consider concurrency. Different from {\pu}, our approach
for sequential settings does not require programs to stop at branches nor use
extra security levels to indicate mutation of labels. Austin and Flanagan
propose a \emph{privatization} operation to boost the permissiveness of {\pu}.
It is not clear how such technique generalizes to arbitrary lattices. Moreover,
the privatization operation can only enforce non-interference when output are
suppressed after branching on a marked flow-sensitive reference. In contrast,
our approach allows changes in flow-sensitive labels, branching on them, and
performing outputs when safe, e.g., a {\toLabeledWith} block follow by writing 
into a reference, which simulates an output in our attacker model. 

%% Breeze
Recently, Hritcu et al.~\citep{10.1109/SP.2013.10} propose a floating-label
system called Breeze. It allows label changes in the current label (i.e. pc) and
consider flow-insensitive objects. Given the design similarities with
\LIO~\citep{stefan:lio}, we believe that our results could be easily adapted to
Breeze.

%% JSFlow
In a web scenario, and focusing on reducing the number of rejected web pages,
Hedin et al. \citep{Hedin13} recently develop JSFlow, a IFC flow-sensitive
monitor for JavaScript. The monitor utilizes the label changing policy
{\nsu}. To overcome some of the restrictions imposed by this discipline, the
primitive \textbf{upgrade} is introduced to explicitly change labels. The
flow-sensitive version of {\forkLIO} can be seen as an automatic application of
\textbf{upgrade} every time that the current label gets raised. Using testing,
Birgisson et al.~\citep{Arnar2012} automatically insert instructions
\textbf{upgrade} in order to reduce the number of runs with possible leaks due
to label changes.



%% OS-like work for the browser 
Coarse-grain IFC enforcements, similar to the ones found in OS work, have been
applied to web browsers. BFlow~\citep{Yip:2009} tracks flows of information at
the granularity of secure zones, i.e., compartments composed of one or several
iframes. The zones' labels (i.e. subjects) must be explicitly updated---no
implicit tainting. The data-confined sandbox (DCS)
system~\citep{conf/esorics/AkhaweLHSS13}, on the other hand, allows implicit
tainting while restricting the propagation of information by using iframes and
mediating crossdomain operations (e.g. access to local storage, fragment-IDs,
network communication, etc.). It is unclear that flow-sensitive objects would be
useful in such scenarios. Naturally, the DOM-tree could be thought as being
composed of flow-sensitive objects, which change security labels based on the
dynamic behavior of the web page~\citep{Russo:2009}. However, the
course-granularity of the mentioned approaches leads to treat the whole DOM-tree
as having the same security label as the security zone (BFlow) or sandbox iframe
(DCS). 

%% Logic / verification approaches 
Hoare-like logics for IFC are often
flow-sensitive~\citep[e.g.][]{Amtoft:2006,Nanevski:2011}. Different from dynamic
approaches, these logics have the ability to observe all the execution paths and
safely approximate label changes. As a result, no leaks due to label changes is
present in provably secure programs.


%% Hybrid approaches
Le Guernic et al.~\citep{LeGuernic:2006,Guernic:2007:ACM} combine dynamic and
static checks in a flow-sensitive execution monitor.
 



% Local Variables:
% TeX-master: "main.ltx"
% TeX-command-default: "Make"
% End:
