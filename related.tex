\section{Related work}
\label{sec:related}

%% Label on the label idea
The \emph{label on the label} could be seen as a fixed label that dictates which
principals can read or modify the policy (inner label) of a flow-sensitive
entity. In a different setting, trust management frameworks have
explored this idea~\cite{4556677}, where role-based rules are labeled to
restrict the view on policies---the mere presence of certain policies could
become inappropriate conduits of information. 

Several authors propose an
\emph{existence security label} to remove leaks due to the termination covert
channel~\cite{DBLP:conf/csfw/RafnssonHS12,Rafnson:2013} or certain behaviors
in dynamic nested data structures~\cite{Russo:2009,DBLP:conf/csfw/HedinS12}. While the
existence security label and the label on the label are structurally isomorphic,
they are used for different purposes and in different scenarios, e.g., it is not
allowed to query labels in
\cite{Russo:2009,DBLP:conf/csfw/HedinS12,DBLP:conf/csfw/RafnssonHS12,Rafnson:2013}.

%% Flow-sensitive monitors 
Hunt and Sands~\citep{Hunt:2006} show the equivalence (modulo code
transformation) between flow-sensitive and flow-insensitive type-systems. 
%Until this work, there was not a similar result concerning purely dynamic
% monitors. 
In a dynamic setting, Russo and Sabelfeld~\citep{Russo:2010} formally pin down
the menace of mutable labels for purely dynamic monitors. They prove that
monitors require static analysis in order to be more permissive than traditional
flow-sensitive type-systems. Targeting purely dynamic monitors, Austin and
Flanagan~\citep{Austin:Flanagan:PLAS09,Austin:Flanagan:PLAS10} describes the
label-change policies \emph{\nsu} (originally proposed by
  Zdancewic~\cite{Zdancewic02programminglanguages}) and \emph{\pu}, where the
latter is provably more permissive than the former, i.e., it rejects fewer
programs. The {\nsu} discipline stops execution on any attempt to change the
label of a public variable inside a secret context. In contrast, {\pu} allows
such changes, marking the altered variables so that the program cannot
subsequently branch on them. The marking consists in replacing the security
label of the variables with |marked|, where |low canFlowTo high canFlowTo
marked|.  Austin and Flanagan propose a \emph{privatization} operation to boost
the permissiveness of {\pu}.  It is not clear how this technique generalizes to
arbitrary lattices. Moreover, the privatization operation can only enforce
non-interference when outputs are suppressed after branching on a marked
flow-sensitive reference. Unfortunately, none of the mentioned work 
consider flow-sensitive in the presence of concurrency.
In fact, the notion of {\pu} does not easily generalize to
the concurrent setting, as this would require tracking occurrences
of branches across threads.

The flow-sensitive extension for LIO is capable of encoding a generalized version
of {\nsu} by using an extra level of indirection.  To illustrate this, we
consider all the references as three-level nested labeled
values---easily expressible in \liofs{}.
Specifically, we use flow-sensitive references of
type |LIORef S (Labeled l a)|, where |mS(a) = LabeledTCB l (LabeledTCB l
(LabeledTCB l' t))| for a given flow-sensitive reference at address |a|. Observe
that the first and second label of the nested structure are the same---this is
effectively like having collapsed the first and second levels of the structure. For
creating a reference, we execute |r <- newRef S lcurr lv|, where |lcurr| is the
current label and |lv| is just a labeled value of the form |LabeledTCB l'' t|
for an arbitrary label |l''|.  The newly created reference is of the form
|LabeledTCB lcurr (LabeledTCB lcurr lv)|.  As in the {\nsu} case, writing a (labeled)
value into r using |writeRef S r lv'|, requires that the current label
flow to |lcurr|, i.e., the label on the label at the time of creating
|r|. Howerver, the {\nsu} rule
prevents a program from modifying a reference |r|, created in a public
environment (|low|), from being modified in a secret context
(|high|). However,
|writeRef S r lv'| allows changing the initially stored labeled value
|lv| with another arbitrary-labeled value |lv'|. Such ``label changes'' (of
the innermost labeled value) make our encoding more permissive than
{\nsu}, which only allows
changing the label of a reference if the new label is 
above the initial label of the reference.


%Different from {\pu}, our approach
%for sequential settings does not require programs to stop at branches nor use
%extra security levels to indicate mutation of labels. 
% Our approach for flow-sensitive references in sequential LIO without the upgrade operation is 
% It is difficult to compare the permissiveness of {\nsu} and {\pu} with our
% sequential approach. In some cases, our mechanism allows changes in
% flow-sensitive labels, branching on them, and performing outputs when
% it is safe to do so. Therefore, it seems to be more permissive.  For instance, consider the
% following piece of code.
% \hspace{-10pt}
% \begin{code}
% do  writeLIORef tmp 0
%     toLabeled' [tmp] H $ 
%               do  x <- readLIORef secret 
%                   when (x > 0) $ writeLIORef tmp (2*x)
%     toLabeled' [tmp] H $ do  
%               do  r <- readLIORef tmp 
%                   when (r > 0) $ writeLIORef secret r
%     writeLIORef public 1
% \end{code}
% The initial label for |tmp| is |low|. Observe that all the runs of this program
% are accepted by our enforcement.  In contrast, the discipline {\nsu} would have
% stopped the code at the line |writeLIORef temp (2*x)| (i.e. when changing the
% label of a |low| reference in a high context). The policy {\pu} would have done
% it at |when (r > 0) $ writeLIORef secret r| (i.e. when branching on the content of
% |temp|).  On the other hand, there are cases where our approach is more
% conservative. Specifically, it might occur that not all the references appearing
% in the list given to |toLabeled'| are modified in every run.  To illustrate that,
% we consider the following piece of code.
% \begin{code}
% do  writeLIORef tmp 0
%     toLabeled' [tmp] H $ readLIORef secret
%     x <- readLIORef tmp 
%     writeLIORef public x 
% \end{code}
% As before, the initial label for |tmp| is |low|. Clearly, every run of this
% program gets rejected by our approach even tough it is secure. Observe that
% after the execution of |toLabeled'|, reference |tmp| changes its label to |high|
% and no more public side-effects are allowed. The discipline {\nsu} and {\pu} do
% not reject this program. If the list given to |toLabeled'| was empty, the program
% above would get  accepted by our approach. 



%% Breeze
Recently, Hritcu et al.~\citep{Breeze} propose a floating-label
system called Breeze. Like LIO, Breeze allows changes in the current
context label (i.e., |pc|) and only considered values with
flow-insensitive labels. Given the design similarities with
LIO~\citep{stefan:lio}, we believe that our results could be easily
adapted to Breeze.

%% JSFlow
Hedin et al. \citep{Hedin13} recently developed JSFlow, an IFC flow-sensitive
monitor for JavaScript. The monitor uses the {\nsu} label changing
policy. To overcome some of the restrictions imposed by this discipline, the
primitive \textbf{upgrade} is introduced to explicitly change labels. Our
upgrade operation resembles that proposed by Hedin et al. Moreover, the
extension to |unlabel| as described Section \ref{sec:flow-sensitive} can be seen as an
automatic application of \textbf{upgrade} every time that the current label gets
raised. Using testing, Birgisson et al.~\citep{Arnar2012} automatically insert
\textbf{upgrade} instructions to boost the permissiveness of {\nsu}.
We further extend this concept of (automatic) \textbf{upgrade}s to a concurrent setting.


The Operating System IFC community has also treated the mutable label
problem in the presence of purely dynamic monitors.
%
Specifically, modern IFC OSes such aas
Asbestos~\cite{Efstathopoulos:2005}, HiStar~\cite{zeldovich:histar},
and Flume~\cite{hkrohn:flume} distinguish between subjects
(processes), and objects ( files, sockets, etc.) such that the
security labels for objects are immutable, while subject labels change
according to the sensitivity of data being read.
%
As in language-based IFC systems, changing the label of subjects and
object can become a covert channel, if not handled appropriated.
%
Hence, HiStar and Flume require that the label of a subject be done
done explicitly by the subject code.
%
Asbestos, on the other hand, allowed (unsafe) changes to labels as the result
of receiving messages under specific and safe conditions.
%
Our work extends on these concepts to allow for changes in object
labels.

%% OS-like work for the browser 
Coarse-grained IFC enforcements, similar to the ones found in IFC OS
work, have been applied to web browsers. BFlow~\citep{Yip:2009} tracks
the flow of information at the granularity of protection zones, i.e.,
compartments composed of iframes. As in LIO, a zone's label, i.e., a
subject's label, must be explicitly updated. Different from LIO, BFlow
does not have finer-grained labeled object; hence the flow-sensitivity
result is only applicable at the protection zone level.  Naturally, by
taking a more fine grained approach, the DOM-tree could be thought of
as being composed of flow-sensitive objects, whose security
labels change according to the dynamic behavior of the web
page~\citep{Russo:2009}.

%% Logic / verification approaches 
Hoare-like logics for IFC are often
flow-sensitive~\citep[e.g.][]{Amtoft:2006,Nanevski:2011}. Different from dynamic
approaches, these logics have the ability to observe all the execution paths and
safely approximate label changes. As a result, no leaks due to label changes are
present in provably secure programs.
%% Hybrid approaches
Le Guernic et al.~\citep{LeGuernic:2006,Guernic:2007:ACM} combine dynamic and
static checks in a flow-sensitive execution monitor.
%
For a flow-sensitive type-system, Foster et
al.~\cite{Foster:2002:FTQ:512529.512531} propose a \textbf{restrict}
primitive that limits the use of variables' aliases in a block of
code. Our |withRefs| is similar to \textbf{restrict} in being used to
increase the permissiveness of the analysis.




 



% Local Variables:
% TeX-master: "main.lhs.tex"
% TeX-command-default: "Make"
% End:
