\section{Related work}
\label{sec:related}

%% Flow-sensitive monitors 
Hunt and Sands~\citep{Hunt:2006} show the equivalence (modulo code
transformation) between flow-sensitive and flow-insensitive type-systems. 
%Until this work, there was not a similar result concerning purely dynamic
% monitors. 
In a dynamic setting, Russo and Sabelfeld~\citep{Russo:2010} formally pin down
the menace of mutable labels for purely dynamic monitors. They prove that
monitors require static analysis in order to be more permissive than traditional
flow-sensitive type-systems. Targeting purely dynamic monitors, Austin and
Flanagan~\citep{Austin:Flanagan:PLAS09,Austin:Flanagan:PLAS10} provide the
label-change policies \emph{\nsu} and \emph{\pu}, where the latter is provably
more permissive than the former (i.e. it rejects fewer suspicious programs). The
{\nsu} discipline stops  execution on any attempt to change the label of a
public variable inside a secret context. In contrast, {\pu} allows such changes,
marking the altered variables so that the program cannot subsequently branch on
them. The marking consists in replacing the security label of the variables with
|marked|, where |low canFlowTo high canFlowTo marked|.  Austin and Flanagan
propose a \emph{privatization} operation to boost the permissiveness of {\pu}.
It is not clear how this technique generalizes to arbitrary lattices. Moreover,
the privatization operation can only enforce non-interference when outputs are
suppressed after branching on a marked flow-sensitive reference. Unfortunately,
none of the mentioned work so far consider concurrency. In fact, the notion of
{\pu} does not easily generalize to the concurrent setting. The reason is that
it would require tracking occurrences of branches across threads, which
challenges scalability.

The flow-sensitive extension for LIO is capable of encoding a generalized version
of {\nsu} by using an extra level of indirection.  To illustrate that, we
consider all the references as three-level nested labeled values---perfectly
expressible in \liofs{}. Specifically, we utilize flow-sensitive references of
type |LIORef S (Labeled l a)|, where |mS(a) = LabeledTCB l (LabeledTCB l
(LabeledTCB l' t))| for a given flow-sensitive reference at address |a|. Observe
that the first and second label of the nested structure are the same---this is
effectively like having collapsed the first and second levels of the structure. For
creating a reference, we call |r <- newRef S lcurr lv|, where |lcurr| is the
current label and |lv| is just a labeled value of the form |LabeledTCB l'' t|
for an arbitrary label |l''|.  The newly created reference is of the form
|LabeledTCB lcurr (LabeledTCB lcurr lv)|.  As in {\nsu}, writing a (labeled)
value into r using |writeRef S r lv'|, demands that the current label flows into
|lcurr|, i.e., the label on the label at the time of creating |r|. This
restriction rules out situations where, if |r| was created in a public
environment (|low|), it would not be possible to modify it in a secret context
(|high|). These are exactly the forbidden label changes in {\nsu}. Additionally,
|writeRef S r lv'| allows changing the initially stored labeled value |lv| for
another one |lv'| with an arbitrary label. This label change (on the last
labeled value of the nested structure) might be rejected by {\nsu}, where it is
only possible to change the label of a reference if the new label is 
above the label given to the reference at the time of creation. In our encoding,
instead, |writeRef S r lv'| always succeeds when the current label is
|lcurr|, regarless of the |labelOf lv'|. 


%Different from {\pu}, our approach
%for sequential settings does not require programs to stop at branches nor use
%extra security levels to indicate mutation of labels. 
% Our approach for flow-sensitive references in sequential LIO without the upgrade operation is 
% It is difficult to compare the permissiveness of {\nsu} and {\pu} with our
% sequential approach. In some cases, our mechanism allows changes in
% flow-sensitive labels, branching on them, and performing outputs when
% it is safe to do so. Therefore, it seems to be more permissive.  For instance, consider the
% following piece of code.
% \hspace{-10pt}
% \begin{code}
% do  writeLIORef tmp 0
%     toLabeled' [tmp] H $ 
%               do  x <- readLIORef secret 
%                   when (x > 0) $ writeLIORef tmp (2*x)
%     toLabeled' [tmp] H $ do  
%               do  r <- readLIORef tmp 
%                   when (r > 0) $ writeLIORef secret r
%     writeLIORef public 1
% \end{code}
% The initial label for |tmp| is |low|. Observe that all the runs of this program
% are accepted by our enforcement.  In contrast, the discipline {\nsu} would have
% stopped the code at the line |writeLIORef temp (2*x)| (i.e. when changing the
% label of a |low| reference in a high context). The policy {\pu} would have done
% it at |when (r > 0) $ writeLIORef secret r| (i.e. when branching on the content of
% |temp|).  On the other hand, there are cases where our approach is more
% conservative. Specifically, it might occur that not all the references appearing
% in the list given to |toLabeled'| are modified in every run.  To illustrate that,
% we consider the following piece of code.
% \begin{code}
% do  writeLIORef tmp 0
%     toLabeled' [tmp] H $ readLIORef secret
%     x <- readLIORef tmp 
%     writeLIORef public x 
% \end{code}
% As before, the initial label for |tmp| is |low|. Clearly, every run of this
% program gets rejected by our approach even tough it is secure. Observe that
% after the execution of |toLabeled'|, reference |tmp| changes its label to |high|
% and no more public side-effects are allowed. The discipline {\nsu} and {\pu} do
% not reject this program. If the list given to |toLabeled'| was empty, the program
% above would get  accepted by our approach. 



%% Breeze
Recently, Hritcu et al.~\citep{10.1109/SP.2013.10} propose a floating-label
system called Breeze. It allows label changes in the current label (i.e. pc) and
considers flow-insensitive values. Given the design similarities with
LIO~\citep{stefan:lio}, we believe that our results could be easily adapted to
Breeze.

%% JSFlow
In a web scenario, and focusing on reducing the number of rejected web pages,
Hedin et al. \citep{Hedin13} recently developed JSFlow, an IFC flow-sensitive
monitor for JavaScript. The monitor utilizes the {\nsu} label changing
policy. To overcome some of the restrictions imposed by this discipline, the
primitive \textbf{upgrade} is introduced to explicitly change labels. Our
upgrade operation resembles that proposed by Hedin et el. Moreover, the
extension to |toLabeled| in Section \ref{sec:flow-sensitive} can be seen as an
automatic application of \textbf{upgrade} every time that the current label gets
raised. Using testing, Birgisson et al.~\citep{Arnar2012} automatically insert
\textbf{upgrade} instructions to boost the permissiveness of {\nsu}. We show how
the concept of upgrade can be extended to work in a concurrent setting.



%% OS-like work for the browser 
Coarse-grained IFC enforcements, similar to the ones found in OS work, have been
applied to web browsers. BFlow~\citep{Yip:2009} tracks flows of information at
the granularity of secure zones, i.e., compartments composed of one or several
iframes. The zones' labels (i.e. subjects) must be explicitly updated---no
implicit tainting. The data-confined sandbox (DCS)
system~\citep{conf/esorics/AkhaweLHSS13}, on the other hand, allows implicit
tainting while restricting the propagation of information by using iframes and
mediating crossdomain operations (e.g. access to local storage, fragment-IDs,
network communication, etc.). It is unclear that flow-sensitive objects would be
useful in such scenarios. Naturally, the DOM-tree could be thought of as being
composed of flow-sensitive objects, which change security labels based on the
dynamic behavior of the web page~\citep{Russo:2009}. However, the
coarse-granularity of the mentioned approaches leads to treating the whole DOM-tree
as having the same security label as the security zone (BFlow) or sandbox iframe
(DCS). 

%% Logic / verification approaches 
Hoare-like logics for IFC are often
flow-sensitive~\citep[e.g.][]{Amtoft:2006,Nanevski:2011}. Different from dynamic
approaches, these logics have the ability to observe all the execution paths and
safely approximate label changes. As a result, no leaks due to label changes are
present in provably secure programs.


%% Hybrid approaches
Le Guernic et al.~\citep{LeGuernic:2006,Guernic:2007:ACM} combine dynamic and
static checks in a flow-sensitive execution monitor.
 



% Local Variables:
% TeX-master: "main.lhs.tex"
% TeX-command-default: "Make"
% End:
