\section{Related work}

\paragraph{Language-based IFC}
Hunt and Sands~\citep{Hunt:2006} show the equivalence (modulo code
transformation) between flow-sensitive and flow-insensitive type-systems. In a
dynamic setting, Russo and Sabelfeld~\citep{Russo:2010} formally pin down the
menace of mutuable label for purely dynamic monitors. They prove that monitors
require static analysis in order to be more permissive than traditional
flow-sensitive type-systems. Targeting purely dynamic monitors, Austin and
Flanagan~\citep{Austin:Flanagan:PLAS10} provide the label changes policies
\emph{no-permissive upgrades} and \emph{permissive upgrades}, where the latter
is provably more permissive (i.e. it rejects less suspicious programs) than the
former. Unfortunately, none of the mentioned work consider
concurrency. Different from permissive upgrades, our approach for sequential
settings does not require programs to stop at branches nor use extra security
levels to indicate mutation of labels. Austin and Flanagan propose a
\emph{privatization} operation to increase the permissiveness of permissive
upgrades. However, it is not clear how such technique generalizes to arbitrary
lattices.
\hl{I assume we explained no-permissive upgrades and permissive upgrades 
before in the paper}



% Local Variables:
% TeX-master: "main.ltx"
% TeX-command-default: "Make"
% End:
