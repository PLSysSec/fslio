\section{Introduction}
\label{sec:intro}

%% Information-flow control as a promising technology 
Information-flow control (IFC) emerges as a promising technology to preserve
confidentiality and integrity of data in presence of untrusted code.  In fact,
it has become particularly attractive for web applications
~\citep[e.g.][]{DeGroef:2012:FWB:2382196.2382275, giffin:hails, yang:2013:towards,
  conf/esorics/AkhaweLHSS13, Hedin13} as well as mobile phone platforms
~\citep[e.g.][]{Enck:2010,android:esorics13}), where the challenge is to allow
untrusted code to access confidential information while restricting its
manipulation.

% Intro to non-interference 
IFC focus on preventing, or limiting, leaks of secret information into public
channels. When no leaks are desired, the chosen baseline policy is
\emph{non-interference}~\citep{Goguen:Meseguer:Noninterference}, and therefore
demanding no dependence of public events on secret data. In scenarios where
non-interference is a strong policy, means for declassification, or intended
release of information, is needed~\citep{Sabelfeld:Sands:CSFW05}. In this
article, we focus on \emph{non-interference}. 

%% Language-based enforcement 
Recently, it has become popular to provide language-based IFC enforcements in
the form of execution monitors~\citep{Hedin2011}. This decision is mainly
motivated by (i) the permissiveness of dynamic techniques over static
ones~\citep{Sabelfeld:Russo:PSI09}, and (ii) the simplicity for treating notions
like dynamic code evaluation---a feature commonly present in modern scripting
languages. To describe confidentiality demands on data, IFC monitors often
maintain a mapping from variables to security
labels~\citep{myers:dlm,Stefan:2011}.  For simplicity, we only consider two
labels: \high (high), which denotes secrets, and \low (low), which denotes
public data. The partial ordering $\low \canflow \high$ and $\high \not \canflow
\low$ expresses that information can only flow from public data into secret
entities. 

%% Treatment of variables 
%% Definition flow-sensitive, flow-insensitive 
One of the facets to IFC analysis lies in the treatment of
variables~\citep{Hunt:2006}. Some analyzes, called \emph{flow-insensitive},
forbid variables to change its security labels at execution time, i.e., labels
are \emph{immutable}. In contrast, \emph{flow-sensitive} enforcements admit
changes on variables' security labels provided that they reflect the
confidentiality of the stored data, i.e., labels are \emph{mutable}. For
instance, assuming secret variable $h$ and function $\mathit{publish}(v)$, which
produce a public event of value $v$, the program $P = h := 0 ;
\mathit{publish}(h)$ is accepted by the analysis since the label associated to
$h$ changed to $L$ after the first assignment.

%Although static flow-sensitive and flow-insensitive analysis are
%equivalent in expressiveness (module code rewriting)~\citep{Hunt:2006}, the 
%situation is not as clear for purely dynamic analysis.

%% Why one instead of the other one 
%%%%% FS+ Permissiveness in dynamic approaches, limited resources, 
%%%%% FI+ Compositionality, good for static analysis 
%On one hand, 
Flow-sensitive analyzes are beneficial for some scenarios. In presence of scarce
resources, for instance CPU registers, it is not feasible to think that they
will be assigned a fixed security label throughout program execution. More
importantly, having flow-sensitivite variables often boost permissiveness since
analyses are able to leverage the changes on labels to avoid rejecting programs
(recall program $P$ above). Unfortunately, purely dynamic IFC systems can turn
the change of labels itself into a covert channel~\citep{Russo:2010}. To solve
this problem, there exists several proposals: incorporate some static analysis
into the execution monitor~\citep{Russo:2010}, forbid certain label changes ---a
policy known as \emph{no-sensitive upgrades}~\citep{Austin:Flanagan:PLAS10}---,
and deny branches on certain variables which have mutated their labels---a
policy named \emph{permissive upgrades}~\citep{Austin:Flanagan:PLAS10}. 

%% Operating system community
Before language-based techniques, the operating system community has
traditionally handle the problems inherent with flow-sensitive analysis and
purely dynamic approaches. Modern examples of that are the operating systems
Asbestos~\citep{Efstathopoulos:2005}, HiStar~\citep{zeldovich:histar}, and
Flume~\citep{krohn:flume}. In a nutshell, those systems are composed of subjects
(e.g. processes) and objects (e.g. files or sockets) with security labels. While
security labels for objects are immutable, labels for subjects can be
changed---often by a more restrictive one. Making label changes implicit as a
result of observing data from object, an action known as \emph{tainting}, turns
changing labels into a covert channel. To close it, HiStar and Flume require
label changes to be done explicitly. Instead, Asbestos allows changes in 
security labels as the result of receiving messages from other subjects  
under safe conditions.

%% LIO bringing things together 


%% Flow sensitivity here





% Local Variables:
% TeX-master: "main.ltx"
% TeX-command-default: "Make"
% End:
