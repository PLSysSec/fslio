\section{Introduction}
\label{sec:intro}

%% Information-flow control as a promising technology 
Information-flow control (IFC) emerges as a promising technology to preserve
confidentiality and integrity of data in presence of untrusted code.  In fact,
it has become particularly attractive for web applications
(e.g.~\citep{DeGroef:2012:FWB:2382196.2382275, giffin:hails, yang:2013:towards,
  conf/esorics/AkhaweLHSS13, Hedin13}) as well as mobile phone platforms
(e.g.~\citep{Enck:2010,android:esorics13}), where the challenge is to allow
untrusted code to access confidential information while restricting its
manipulation.

% Intro to non-interference 
IFC focus on preventing, or limiting, leaks of secret information into public
channels. When no leaks are desired, the chosen baseline policy is
\emph{non-interference}~\citep{Goguen:Meseguer:Noninterference}, and therefore
demanding no dependence of public events on secret data. In scenarios where
non-interference is a strong policy, means for declassification, or intended
release of information, is needed~\citep{Sabelfeld:Sands:CSFW05}. In this
article, we focus on \emph{non-interference}. 

%% Language-based enforcement 
Recently, it has become popular to provide language-based IFC enforcements in
the form of execution monitors~\citep{Hedin2011}. This decision is mainly
motivated by (i) the permissiveness of dynamic techniques over static
ones~\citep{Sabelfeld:Russo:PSI09}, and (ii) the simplicity for treating notions
like dynamic code evaluation---a feature commonly present in modern scripting
languages. To describe confidentiality demands on data, IFC monitors often
maintain a mapping from variables to security
labels~\citep{myers:dlm,Stefan:2011}.  For simplicity, we only consider two
labels: \high (high), which denotes secrets, and \low (low), which denotes
public data. The partial ordering $\low \canflow \high$ and $\high \not \canflow
\low$ expresses that information can only flow from public data into secret
entities. 

%% Treatment of variables 
%% Definition flow-sensitive, flow-insensitive 
%% Why one instead of the other one 
%%%%% FS+ Permissiveness in dynamic approaches, limited resources, less burnen to
%%%%% the programmer 
%%%%% FI+ Compositionality, good for static analysis 
One of the facets to IFC analysis lies in the treatment of variables. Some
analyzes, called \emph{flow-insensitive}, forbid variables to change its
security labels at execution time, i.e., labels are \emph{immutable}. In
contrast, \emph{flow-sensitive} enforcements admit changes on variables'
security labels provided that they reflect the confidentiality of the stored
data, i.e., labels are \emph{mutable}. For instance, assuming secret variable
$h$ and function $\mathit{publish}(v)$, which produce a public event of value
$v$, the program $h := 0 ; \mathit{publish}(h)$ is accepted by the analysis
since the label associated to $h$ changed to $L$ after the first
assignment. Although static flow-sensitive and flow-insensitive analysis are
equivalent in expressiveness (module code rewriting)~\citep{Hunt:2006}, the 
situation is not as clear for  dynamic analysis.

% Flow-insensitive 
% analysis are often popular on static IFC enforcements
% (e.g.~\citep{Volpano:Smith:Irvine:Sound,jif,Pottier:Simonet:POPL02}), where 
% the security of the programs compositionality 
% Flow-sensitivity popular on dynamic 
%,Guernic:2007:ACM,Askarov:2009}).



%% Flow-insensitive 

%% Flow sensitivity 

%% Floating label systems (O.S. people)

%% Flow sensitivity here





% Local Variables:
% TeX-master: "main.ltx"
% TeX-command-default: "Make"
% End:
