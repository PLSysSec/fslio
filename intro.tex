\section{Introduction}
\label{sec:intro}

%% Information-flow control as a promising technology 
Information-flow control (IFC) emerges as a promising technology to preserve
confidentiality and integrity of data in presence of untrusted code.  In fact,
it has become particularly attractive for web applications
(e.g.~\citep{DeGroef:2012:FWB:2382196.2382275, giffin:hails, yang:2013:towards,
  conf/esorics/AkhaweLHSS13, Hedin13}) as well as mobile phone platforms
(e.g.~\citep{Enck:2010,android:esorics13}), where the challenge is to allow
untrusted code to access confidential information while restricting its
manipulation.

% Intro to non-interference 
IFC focus on preventing, or limiting, leaks of secret information into public
channels. When no leaks are desired, the chosen baseline policy is
\emph{non-interference}~\citep{Goguen:Meseguer:Noninterference}, and therefore
demanding no dependence of public events on secret data. In scenarios where
non-interference is a strong policy, means for declassification, or intended
release of information, is needed~\citep{Sabelfeld:Sands:CSFW05}. In this
article, we focus on \emph{non-interference}. 

%% Language-based enforcement 
Recently, it has become popular that language-based IFC enforcements are given
in the form of an execution monitors~\citep{Hedin2011}. This decision is mainly
motivated by (i) the permissiveness of dynamic techniques over static
ones~\citep{Sabelfeld:Russo:PSI09}, and (ii) the simplicity for treating notions
like dynamic code evaluation---a feature commonly present in modern scripting
languages.
%% Treatment of variables 

%% Flow-insensitive 

%% Flow sensitivity 

%% Floating label systems (O.S. people)

%% Flow sensitivity here





% Local Variables:
% TeX-master: "main.ltx"
% TeX-command-default: "Make"
% End:
