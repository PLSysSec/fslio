\section{Introduction}
\label{sec:intro}

Modern software systems are composed of many complex components that
handle sensitive data.
%
In many cases (e.g., mobile and web---both client- and
server-side---applications) these disparate components are provided
by different authors, of varying trustworthiness.
%
This combination of untrusted code and sensitive data 
increases the risk of data theft or corruption.

%% Information-flow control as a promising technology 
Information-flow control (IFC) is a promising approach to security
that provides data confidentiality and integrity in the presence of
untrusted code.
%
In its simplest form, IFC tracks and controls the flow of information
through a system according to a security policy, usually
\emph{non-interference}~\cite{Goguen:Meseguer:Noninterference}.
%
Non-interference states that public events should not depend on
sensitive data and dually, trusted data should not be affected by
untrusted events.
%
Hence, a program, which may be composed of untrusted components, is
guaranteed to preserve data confidentiality and integrity if it
satisfies non-interference.
%
Indeed, this appealing guarantee of IFC has recently led to significant
research and development efforts %that use IFC 
to secure web
applications (e.g.,~\cite{DeGroef:2012:FWB:2382196.2382275,
giffin:hails, yang:2013:towards, Hedin13}) and mobile platforms
(e.g.,~\cite{Enck:2010,android:esorics13}), where untrusted code is
the norm.

%% Language-based enforcement 
In particular, language-based systems that employ dynamic execution
monitors to enforce IFC have become popular~\cite{Hedin2011}.
%
This is due, in part, to the permissiveness of dynamic techniques
(when compared to static approaches~\cite{Sabelfeld:Russo:PSI09}), 
%the ability to specify fine-grained policies, 
%Ale: Type-system handles fine-grained policies too 
 and the ability to handle complex language features like dynamic code evaluation---a feature
common to many scripting languages.
 
To ensure data confidentiality and integrity, these dynamic IFC
systems associate \emph{security labels} with data (e.g., by mapping
variables to labels) and monitor where such data can
flow~\cite{myers:dlm, Stefan:2011}.  
%
In this paper, we use the labels |high| and |low|, to respectively
denote secret and public data, and ensure that information cannot flow
form a secret entity into a public one, i.e., the labels are ordered
such that |low canFlowTo high| and |high cantFlowTo low|.
%
in general, the partial order |canFlowTo| (label check) is used when
governing the allowed flows.
%
We remark that our main results use a generalized lattice that may
 also express integrity concerns~\cite{myers:dlm, Stefan:2011}, we
 only use the two-point lattice for simplicity of exposition.
 
One of the facets of IFC analysis lies in how such labels, when associated with
objects, are treated~\cite{Hunt:2006}.
%
Specifically, some IFC systems (e.g.,~\cite{Breeze, Aeolus, stefan:lio,
stefan:addressing-covert, zeldovich:histar, Efstathopoulos:2005, krohn:flume})
treat labels on objects as \emph{immutable} and do not allow for changes over
the lifetime of the program, i.e., labels of objects are \emph{flow-insensitive}.
%
% In contrast, other systems (e.g.,~\cite{jif, FlowCaml,
% Ale: FlowCaml and JIF are flow-insensitive (they have label inference tough)
In contrast, other systems (e.g.,~\cite{Zdancewic02programminglanguages, Austin:Flanagan:PLAS09,
Austin:Flanagan:PLAS10}) are \emph{flow-sensitive}, i.e., they allow object
labels to change, in certain conditions, according to the sensitivity of the
data that is stored in the object.
%
In general, these flow-sensitive systems are more permissive, i.e., they allow
programs that flow-insensitive monitors would reject.
 
Consider, for instance, a web application that writes to a labeled
log while servicing user requests.
%
If the label of the log is |low|, a flow-insensitive IFC monitor would disallow
writing any sensitive data (e.g., error messages containing user-supplied data)
to the log, since this would constitute a leak.
%
However, in a flow-sensitive system, the label of the log can change (to
|high|), as to accommodate the kinds of data being written to the log.
%
For many applications, allowing labels to change in such a way is very
desirable---it alleviates the burden of having to, a priori, determine the
precise labels of objects (e.g., the log).

Unfortunately, naively introducing flow-sensitive objects to a purely dynamic
IFC system can turn label changes into a covert channel~\cite{Russo:2010}. 
%
Consider the code fragment of Figure~\ref{fig:fsattackintro}
where references |l| and |h| are respectively labeled |low| and
|high|.
%
\begin{wrapfigure}{r}{0.5\columnwidth}
%\vspace{-15pt}
  \small
\begin{code}
l    := true
tmp  := false
if h then tmp := true
if not tmp then l := false
\end{code}
\vspace{-15pt}
%%|l := 1 ; t := 0 ;| \\
%%|if h then t := 1;| \\
%%|if t /= 1 then l := 0| \\
%%|publish(l)| 
\caption{\small\label{fig:fsattackintro} Flow-sensitive attack}
\end{wrapfigure}
%
By naively allowing arbitrary label changes---even if the new label is
more sensitive---we can leak the contents of |h| into |l|.
%
In particular, suppose that the temporary variable |tmp| is initially
labeled |low|.
%
If the value stored in |h| is |true|, then in the first conditional we
assign |true| into |tmp| and raise its label to |high|, reflecting the
fact that the branch condition depends on sensitive data.
%
Since the |tmp| is |true|, the branch condition for the second conditional is
|false| and thus the value and label of |l| are left intact, i.e., |true| at
|low|.
%
However, if |h| is |false|, then the value and label of |tmp| remains
untouched---the code does not perform the first assignment.
%
Instead, the second assignment, setting |l| to |false|, is performed;
importantly, however, the label of |l| remains |low|, since the label of the
branch condition is also |low|.
%
Note that in both cases the label of |l| stays |low|, but the value of |l| is
the same as the secret |h|.
%
In systems such as LIO and Breeze, which allow labels to be inspected, this
attack can be further simplified by simply checking the label of |tmp| after
the first assignment---if the secret is true then the label will be |high|,
otherwise it will be |low| (hence why the \emph{label change} is considered a
covert channel).

This attack is not new, and, to ensure that the covert channel is
not introduced when adding flow-sensitive references in such a way,
several solutions have already been proposed. 
%
These solutions fall into roughly three categories.
%
First, the IFC monitor can incorporate static information to ensure that such
leaks are disallowed~\cite{Russo:2010}.
%
Second, the IFC monitor can forbid certain label changes, depending on the
context (e.g., the program counter (\emph{pc})
label~\cite{sabelfeld:language-based-iflow}).
%
For instance, the \emph{no-sensitive upgrades} policy disallows raising the
label of a public reference in a sensitive context (e.g., when a branch
condition is |high|)~\cite{Zdancewic02programminglanguages,
Austin:Flanagan:PLAS09}.
%
And, third, the monitor can disallow branches that depend on certain variables,
for which the label was mutated, as done by the \emph{permissive upgrades}
policy~\cite{Austin:Flanagan:PLAS10}.

In this paper, we take a fresh perspective on flow-sensitivity %the no-sensitive upgrades
% discipline 
in the context of coarse-grained floating-label systems, in
particular, the LIO IFC system~\cite{stefan:lio,
stefan:addressing-covert}. % Ale: remove Haskell  
%
LIO brings ideas from IFC Operating Systems---notably,
HiStar~\cite{zeldovich:histar}, Asbestos~\cite{Efstathopoulos:2005}, and
Flume~\cite{krohn:flume}---into a language-based setting.
%
In particular, LIO takes an OS-like coarse-grained approach by associating a
single ``current'' floating-label with a computation (and everything in scope),
instead of heterogeneously labeling every variable, as typically done by
language-based systems (e.g.,~\cite{jif,FlowCaml}).
%
This floating-label is raised (e.g., from |low| to |high|) to accommodate
reading sensitive data and thus serves as a form of ``taint'' reflecting the
sensitive of data in context, i.e., LIO is flow-sensitive in the current label.
%
(This can be seen as raising the \emph{pc} in more traditional language-based
systems.)
%
In turn, the LIO monitor uses the ``current'' floating-label to
restrict where the computation can write (e.g., once the current label
is raised to |high| it can no longer write to references labeled
|low|).
%
However, and like other similar IFC systems, LIO is \emph{flow-insensitive} in
object labels.

%% Contributions 
%Inspired by previous results~\citep{Austin:Flanagan:PLAS10}, 
This work extends the LIO IFC system, both the sequential and concurrent
versions, to incorporate flow-sensitive references.
%
A key insight of this work is to consider labels of references as
being composed of two elements: the reference label describing the
confidentiality of the stored value, and another label, called
\emph{the label on the label}, which describes the confidentiality of
the reference label itself.
%
Our monitor, then only forbids changing a label of a reference if
\emph{the label on the label is below the floating-label}.
%
Inspired by~\cite{Hedin13}, we add a primitive for \emph{upgrading} labels,
when permitted by our monitor.
%
This boosts the permissiveness of LIO, and, for instance, allows programs, such
at the logging web application described above, which would otherwise be
rejected by the IFC monitor.

To reduce the programmer's burden of introducing upgrade annotations, our
calculus provides a means for automatically upgrading references for which the
computation is about to ``lose'' write access, i.e., before tainting the
computation by raising the current label, we first upgrade all the references
that are less sensitive than this label.
%
While secure, this feature facilitates a form of \emph{label creep},
wherein all flow-sensitive references might end up with labels that
are ``too high.''
%
To further address this, we propose a block-structured primitive which
only upgrades the labels of declared flow-sensitive references, while
disallowing access to undeclared ones.
 
By taking a fresh perspective on flow-sensitivity, we also show that the {\nsu}
policy and our upgrade operation can be encoded using existing flow-insensitive
constructs---the key insight is to explicitly model labels on labels with
\emph{nested labeled references}.
%
In the context of LIO, this encoding has the added benefit of allowing us to prove
non-interference by simply invoking previous results.
%
More interestingly, our sequential semantics for LIO with flow-sensitive
references directly extend to the concurrent setting.
 
The contributions of this paper are as follows:
\begin{itemize}

\item We extend LIO to incorporate flow-sensitive objects, with a focus on
references. This extension not only increases LIO's permissiveness, but also provides a
means for safely combining flow-insensitive and flow-sensitive references.

\item We present a uniform mechanism for treating flow-insensitive and
flow-sensitive references in both sequential and concurrent settings. To the
best of our knowledge, we are the first to analyze the challenges of purely
dynamic monitors with flow-sensitive references in the presence of concurrency.

\item A non-interference proof for the different calculi that leverages the
encoding of flow-sensitive references using flow-insensitive constructs.
\end{itemize}
%
We remark that while our development focuses on LIO, we believe that
our results generalize to other sequential and concurrent
floating-label systems (e.g.,~\cite{Breeze, Efstathopoulos:2005,
zeldovich:histar, krohn:flume}).

%\hl{Structure of the paper, i.e., the paper is organized as follows}
The rest of the paper is organized as follows. Section~\ref{sec:background}
provides an introduction to LIO and its formalization.
Section~\ref{sec:flow-sensitive} presents our flow-sensitivity extensions and
enforcement mechanism. Section~\ref{sec:conc} extends this approach to the
concurrent setting. Section~\ref{sec:soundness} presents the embedding of our
enforcement using flow-insensitive constructs, from which our formal security
guarantees follow.  We discuss related work in Section~\ref{sec:related} and
conclude in Section~\ref{sec:conclusion}.


% Local Variables:
% TeX-master: "main.tex"
% TeX-command-default: "Make"
% End:
