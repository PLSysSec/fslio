\appendices
\section{Semantics for the base calculus}
\label{sec:app:sem}

The reduction rules for pure and monadic terms are given in
Figure~\ref{fig:sos:rules}.

\concept{pure red}
%
The reduction rules for pure terms are mostly standard.
%
Only two points are worth noting.
%
First, we define substitution |{t2 / x } t1| in the usual way:
homomorphic on all operators and renaming bound names to avoid
captures. 
%
Second, our label operations |lub|, |glb|, and |canFlowTo| rely on the
label-specific implementation of these lattice operators, as used in
the premise of rule \ruleref{labelOp};
%
we use the meta-level partial function |denot cdot|, which maps terms
to values, to precisely capture this implementation detail.

\concept{monad red}
%
The reduction rules for monadic terms deserve some attention.
%
As a result of this encoding, the definition for |return| and |(>>=)|
are trivial: the former simply reduces to a monadic value by wrapping
the term with the |LIOTCB| constructor, while the latter evaluates the
left-hand term and supplies the result to the right-hand term, as
usual.


\begin{figure}[t] % sos:rules
\small
\begin{code}
Ep  ::= Ep t | fix Ep | if Ep then t else t | Ep lop t | v lop Ep
E   ::= []| Ep | E >>= t 
\end{code}

\begin{mathpar}
\inferrule[app]
{ } { |Ep[(\x.t1) t2] ~> Ep[{ t2 / x } t1]| }
\and
\inferrule[fix]
{ } { |Ep[fix (\x.t)] ~> Ep[{fix (\x.t) / x } t|] }
\and
\inferrule[ifTrue]
{ } { |Ep[if true then t2 else t3] ~> Ep[t2]| }
\and
\inferrule[ifFalse]
{ } { |Ep[if false then t2 else t3] ~> Ep[t3]| }
\and
\inferrule[labelOp]
{ | v = denot ( l1 lop l2 ) | }
{ |Ep[l1 lop l2] ~> Ep[v]| }
\and
\inferrule[return]
{ } { |conf c (E[return t]) --> conf c (E[LIOTCB t])| }
\and
\inferrule[bind]
{ }
{ |conf c (E[(LIOTCB t1) >>= t2]) --> conf c (E[t2 t1])| }
\and
\inferrule[getLabel]
{ |c = (lcurr, ...)|  }
{
|conf c (E[getLabel]) --> conf c (E[return lcurr])|
}
\end{mathpar}
\caption{Reduction rules for base \lio.\label{fig:sos:rules}}
\end{figure}