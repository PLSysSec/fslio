\appendices
\section{Semantics for the base calculus}
\label{sec:app:sem}

\begin{figure}[t] % sos:rules
\small
\begin{code}
Ep  ::= Ep t | fix Ep | if Ep then t else t | Ep lop t | v lop Ep
E   ::= []| Ep | E >>= t 
\end{code}

\begin{mathpar}
\inferrule[app]
{ } { |Ep[(\x.t1) t2] ~> Ep[{ t2 / x } t1]| }
\and
\inferrule[fix]
{ } { |Ep[fix (\x.t)] ~> Ep[{fix (\x.t) / x } t|] }
\and
\inferrule[ifTrue]
{ } { |Ep[if true then t2 else t3] ~> Ep[t2]| }
\and
\inferrule[ifFalse]
{ } { |Ep[if false then t2 else t3] ~> Ep[t3]| }
\and
\inferrule[labelOp]
{ | v = denot ( l1 lop l2 ) | }
{ |Ep[l1 lop l2] ~> Ep[v]| }
\and
\inferrule[return]
{ } { |conf c (E[return t]) --> conf c (E[LIOTCB t])| }
\and
\inferrule[bind]
{ }
{ |conf c (E[(LIOTCB t1) >>= t2]) --> conf c (E[t2 t1])| }
\and
\inferrule[getLabel]
{ |c = (lcurr, ...)|  }
{
|conf c (E[getLabel]) --> conf c (E[return lcurr])|
}
\end{mathpar}
\caption{Reduction rules for base \lio.\label{fig:sos:rules}}
\end{figure}


The reduction rules for pure and monadic terms are given in
Figure~\ref{fig:sos:rules}. 
% 
We define substitution |{t2 / x } t1| in the
usual way: homomorphic on all operators and renaming bound names to avoid
captures.
%
Our label operations |lub|, |glb|, and |canFlowTo| rely on the label-specific
implementation of these lattice operators, as used in the premise of rule
\ruleref{labelOp}; we use the meta-level partial function |denot cdot|, which
maps terms to values, to precisely capture this implementation detail.

%\concept{pure red}
%
The reduction rules for pure terms are mostly standard. For instance, 
in rule \ruleref{ifTrue}, when the branch has a true condition, i.e., 
|Ep[if true then t2 else t3]|, it reduces to the then branch (|Ep[t2]|). 
The rest are self-explanatory and we do not discuss them any further. 
%

%\concept{monad red}
%
The reduction rules for monadic terms deserve some attention.
%
The definition for |return| and |(>>=)| are trivial: the former simply reduces
to a monadic value by wrapping the term with the |LIOTCB| constructor, while the
latter evaluates the left-hand term and supplies the result to the right-hand
term, as usual.


\section{attack on naive flow-sensitive references}
\label{sec:app:fs-attack2}
\begin{figure}
\small
\begin{code}
leakRef :: LIORefTCB space Bool -> LIO Bool
leakRef href = do
  lref  <- newRef space low true
  tmp   <- newRef space low false
  toLabeled high $ do  h <- readRef space href
                       when h (writeRef space tmp true)
  toLabeled high $ do  t <- readRef space tmp
                       when (not t) (writeRef space lref false)
  readRef space lref
\end{code}
\caption{An attack in LIO with naive flow-sensitive reference
extension without |labelOfR space|.\label{fig:fs-attack2}}
\end{figure}
%
As in the attack of Figure~\ref{fig:fs-attack}, |leakRef| can be used to leak
the value stores in a |high| reference |href|, while keeping the current label
|low|.
%
Internally, the value is leaked into public reference |lref| by leveraging the
fact that based on a secret value the label of a public reference (|tmp|) can
be changed (or not).
%
In the first |toLabeled| block, if |h==true|, then the label of |tmp| is raised
to |high| and its value is set to |true|.
%
In the second |toLabeled| block, we read |tmp|, which may raise the current
label to |high| if the secret is |true| (and thus |tmp| was upgraded).
%
Indeed, if the secret is |true| (and thus |t==true|) we leave the public
reference intact: |true|.
%
However, if the secret is |false|, the |tmp| reference is not modified in the
first |toLabeled| block and thus when reading it in the second |toLabeled|
block, the current label remains |low|, and since |t==false|, we write |false|
into the public reference.
%
In both cases the value stored in |lref| corresponds to that of |href|, yet
leaving the current label and the label of |lref| intact, |low|---a direct
leak!
