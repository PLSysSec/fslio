\documentclass[10pt, conference, compsocconf, letterpaper]{IEEEtran}
\setlength{\textfloatsep}{0.40cm}
%include polycode.fmt

%format >*> = "\succ"
%format Conc = "Thread"
%format RTSl = "\Sigma.\lbl"
%format RTSl' = "\Sigma'.\lbl"
%format hole = "\bullet"
%format eN = "\eraseName_L"
%format v_s
%format v_RTSl
%format m_1
%format m_2
%format r_s
%format ts = t"_s"
%format t_s
%format e_s
%format elm = "\el{m}"
%format elt_s = "\el{t_s}"
%format ele = "\el{e}"
%format elc = "\el{c}"
%format elp = "\el{p}"
%format sch = "\mathbf{sch}"
%format vss = "v_s"
%format run  = "run"
%format val  = "x"

\usepackage{paralist, balance}
\usepackage{graphicx,subfig,wrapfig}
\usepackage{amsmath,amssymb}%,stmaryrd}
\usepackage{tikz}
\newcommand{\hcancel}[1]{%
    \tikz[baseline=(tocancel.base)]{
        \node[inner sep=0pt,outer sep=0pt] (tocancel) {#1};
        \draw[red, thick] (tocancel.south west) -- (tocancel.north east);
    }%
}%

%\usepackage{mathpartir}
%\usepackage[usenames,dvipsnames]{color}
\usepackage{url}
\usepackage{mathpartir}
\usepackage[utf8]{inputenc}
\newcommand{\ignore}[1]{}
\newcommand{\Red}[1]{{\color{red} #1}}
\newcommand{\Blue}[1]{{\color{blue} #1}}
\newcommand{\concept}[1]{}%{\Blue{\bf #1:}}}
\long\def\comment#1{}
\newcommand{\tocite}[1]{\Red{[cite]}}
\newcommand{\step}{\ensuremath{\triangleright}}
\long\def\cut#1{}

\definecolor{gray}{RGB}{84,84,84}
\definecolor{dark-green}{RGB}{ 0,100,  0}

\newif\ifextended
%\extendedtrue
\extendedfalse

\ifextended
\newcommand{\inputproof}[1]{\input{#1}}
\newcommand{\inlong}[1]{\Red{#1}}
\else
\newcommand{\inputproof}[1]{}
\newcommand{\inlong}[1]{}
\fi

\usepackage{soul} % for highlighting
\usepackage[square,comma,numbers,sort&compress,sectionbib]{natbib}
%\usepackage[square,comma,sectionbib]{natbib}

\newcommand{\nsu}{{no-sensitive-upgrade}}
\newcommand{\pu}{{permissive-upgrade}}

%include notation.jfp.tex
\usepackage{thmtools}
\declaretheorem{theorem}
\declaretheorem[name=Lemma, style=theorem]{lemma}
\declaretheorem[name=Corollary, style=theorem]{corollary}
\declaretheorem[name=Definition, style=definition]{definition}
\declaretheorem[name=Example, style=definition]{example}
% thmstyle: remark
\declaretheorem[name=Proof sketch, style=remark]{proofsketch}

\begin{document}

\title{On Dynamic Flow-sensitive Floating-Label Systems}
\ifextended
\subtitle{Extended Version}

\fi

\author{
\IEEEauthorblockN{Pablo Buiras}
\IEEEauthorblockA{Chalmers \\
    buiras@@chalmers.se}
\and
\IEEEauthorblockN{Deian Stefan}
\IEEEauthorblockA{Stanford \\
    deian@@cs.stanford.edu}
\and
\IEEEauthorblockN{Alejandro Russo}
\IEEEauthorblockA{Chalmers \\ 
    russo@@chalmers.se}
}


\maketitle

%include abstract.tex

%include intro.tex
% 1 1/2 Ale 

%include background.tex
% 2 1/2 Pages (Deian) explain LIO with semantics, style like in JFP

%include flow-sensitive.tex
% 2 Pages (Pablo/Ale) explain sequential 

%include concurrency.tex 
% 1 Page (Pablo/Ale), attack and discussion 

%include implementation.tex
  % 1 Page (Pablo/Ale)

%include soundness.tex
% 2 Pages (Pablo/Ale)

%include related.tex
% 1 Page (Ale) 

%include conclusion.tex
% 1/2 Page Total so far: 11 1/2




\section*{Acknowledgments}
We thank our colleagues in the ProSec group at Chalmers, Stefan Heule,
David Mazi{\`e}res, and Edward Z. Yang for the useful discussions.
%
We thank the anonymous reviewers for constructive feedback on an
earlier version of this work.
%
This work was funded by DARPA CRASH under contract \#N66001-10-2-4088
and the Swedish research agency VR.
%
Deian Stefan is supported by the DoD through the NDSEG Fellowship
Program.

{\frenchspacing
\bibliographystyle{plain}
\bibliography{conferences,dm,local}
}
\balance

%include appendix.tex




\end{document}

% Local Variables:
% TeX-master: "main.ltx"
% TeX-command-default: "Make"
% End:
 
