\section{Concurrency}

In this section, we analyze our flow-sensitivity extensions in presence of
concurrency. We firstly give a brief introduction to a subset of the concurrent
version of |LIO|.
% which includes operations for creating threads and expressing
% communication among them.

Concurrent |LIO|~\cite{?} extends the sequential version of |LIO| with
operations for creating threads and expressing communication among
them, as seen in Figure~\ref{fig:conc-syntax}.
\begin{figure}[ht] % labeled
\small
\begin{code}
v    ::= cdots  | ResultTCB t 
t    ::= cdots  | forkLIO l t | waitLIO t
                | newLMVar l t | takeLMVar t | putLMVar t t
tau  ::= cdots  | Result l tau | LMVar l tau
\end{code}
\caption{Syntax for concurrent |LIO|.\label{fig:conc-syntax}}
\end{figure}

|LMVar|s are an IFC-aware version of |MVar|s, which are
well-established primitives for synchronization in
Haskell~\cite{CH96}.  A value of type |LMVar l a| is a mutable
location that is either empty or contains a value of type |a| labeled
with |l|. They are created with |newLMVar|, and they are accessed with
|takeLMVar| and |putLMVar|. Function |takeLMVar| raises the current
label and empties an |LMVar| if it is full, and blocks
otherwise. Function |putLMVar| fills an |LMVar| with a value if it is
empty and the current label flows to the label of the |LMVar|, and
blocks otherwise. These are the only synchronization and communication
primitives for concurrent |LIO|. For the sake of brevity, we elide a
full presentation of the semantics of |LMVars| since it is standard.

The expression |forkLIO l t| forks a new thread to perform the
computation |t|, where |l| is an upper bound on the current label of
the computation in the spawned thread. Essentially, |forkLIO| is a
concurrent version of |toLabeled|. When a thread is spawned, the
parent receives a value of type |Result l a|, which serves as a
handler that allows to access the labeled result produced by |t|. The
function |waitLIO| allows threads to inspect these values, i.e. the
values returned by spawned computations, in a manner similar to
|unlabel|. After calling |waitLIO|, the current label is raised to the
label of its argument, and the program waits for the target thread to
produce a result, which is then returned. As shown in~\cite{?}, these
primitives serve to close the termination channel, since in order to
observe (non-)termination of a forked thread, we must first raise our
current label. The semantics for these operations are shown in
Figure~\ref{fig:conc-semantics}. Note that we rely on the
implementation of |LMVar|s in order to implement these
operations. Every thread writes its result to an |LMVar|, from which
other threads can read using |waitLIO|.

\begin{figure}[ht]
\begin{mathpar}
  \inferrule[(forkLIO)]
  {|c = (lcurr, mI, mS)| \\
   |lcurr canFlowTo l ~> True| \\
   |c' = c[ mI mapsto mI[ mv mapsto LabeledTCB l boxdot ] ]| \\
   |e = t >>= \x -> putLMVar mv x| \\
   |fresh(mv)|
}
{
|conf c (E[forkLIO l t]) .-->  conf (c') (E[return (ResultTCB mv)])|
}
\and
\inferrule[(waitLIO)]
{}
{|conf c (E[waitLIO (ResultTCB mv)]) --> conf c (E[takeLMVar mv])|}
\end{mathpar}
  \caption{Semantics for |forkLIO| and |waitLIO|.}
  \label{fig:conc-semantics}
\end{figure}

\subsection{Flow-sensitive references}

The results from previous sections extend naturally to the concurrent
setting. However, under the \emph{auto-upgrade} semantics, having a
store which is shared among all threads implies that all references
get upgraded whenever any thread raises its current label. As this
might be too restrictive in practice, we offer |forkLIO'|, an
alternative version of |forkLIO| with an additional parameter which
represents the list of flow-sensitive references into which the thread
is allowed to write. This function is analogous to |toLabeled'| in
that it restricts the scope of the store available to the spawned
thread. Figure~\ref{fig:forkS-semantics} shows the semantics for this
operation. The |addrs| operation is the same as in |toLabeled'| from
the previous section.

\begin{figure}[ht]
  \centering
  \begin{mathpar}
\inferrule[forkLIO']
{|c = (lcurr, mI, mS)|\\
 |lcurr canFlowTo l ~> true|\\
 |mS' = {a mapsto mS(a) inlinesep a element dom mS and addrs(v) }|\\
 |c' = c[ mI mapsto mI[ mv mapsto LabeledTCB l boxdot ] ]| \\
 |e = t >>= \x -> putLMVar mv x| \\
 |fresh(mv)|
}
{|conf c (E[forkLIO' l v t]) .-.-> conf c' (E[return (ResultTCB mv)])|}
    
  \end{mathpar}
  \caption{Semantics for the |forkLIO'| primitive.}
  \label{fig:forkS-semantics}
\end{figure}


% In a sequential setting, the attack in Figure~\ref{fig:fs-attack} is disabled by
% the enforcement presented in the previous section, allowing us to prove
% termination-insensitive non-interference. However, as described before, the
% concurrent version of |LIO| requires the property of
% \emph{termination-sensitive} non-interference. In this section, we show ohow
% threads and label changes on shared references jeopardizes security.  
% Specifically, we present how to exploit the termination channel in concurrent 
% |LIO|.

% \begin{figure}
% \begin{code}
% do  forkLIO H $ -- Thread A
%        do  v <- readLIORef secretM
%           when v (readLIORef secretH)
%     forkLIO H $ -- Thread B
%        do  upgrade shared_ref M
%            writeLIORef public True
% \end{code}  
% \caption{Flow-sensitive attack in a concurrent setting.\label{fig:fs-conc-attack}}
% \end{figure}
% Figure~\ref{fig:fs-conc-attack} shows the attack. This piece of code illustrates
% that our approach for sequential |LIO| does not naturally generalize for
% concurrency.  
% %how to exploit the termination
% %channel to leak information in this context. 
% We assume that threads are interleaved so that thread A's code runs to
% completion before thread B's code. Otherwise, it is enough to insert dummy
% instructions before forking thread B. We also assume the standard three-point
% lattice |low canFlowTo medium canFlowTo high|. Thread A firstly read a secret at
% label |medium|, thereby upgrading |shared_ref| to that label. Depending on the
% read value, the thread then reads a secret with label |high|, which further
% upgrades |shared_ref| to |high|. Thread B, on the other hand, firstly try to
% upgrade |shared_ref| to level |medium|. It is this instruction which exploits
% the termination channel. If |secretM| was |False|, the upgrade operation
% succeeds since |share_ref| is already at level |medium|. In contrast, if
% |secretM| was |True|, the instruction fails due to |shared_ref| having the label
% |high|. This implies that the operation that follows the upgrade instruction,
% which writes a value to a public reference at level |low| (|writeLIORef public
% True|), will either be executed or not depending on a value at level |medium|.
% The fact of stopping thread B at different execution points, depending on
% |secretM|, constitutes a termination leak. 

% In general terms, the attack shows that threads, with different floating-labels,
% might refer to the same shared flow-sensitive reference. If the reference is
% upgraded in one thread based on some confidential data, then it might taint the
% other ones when the reference is read. Such taints might then stop the execution
% of the threads, which consequently could exploit the termination covert channel.

% In order to address this attack, we can either remove shared state altogether,
% or use the \emph{no-sensitive-upgrades} policy without explicit or automatic
% upgrades. At the semantics level, restricting sharing is achieved when forking.

% \hl{Pablo, can you put the rule here avoiding sharing?}

% Once sharing is restricted, it is safe to upgrade flow-sensitive references as 
% we did for the sequential setting of LIO. 

% \hl{Ale: Not sure about the next paragraph, is it the automatically upgrade what 
% does not work? I believe it is the upgrade, so no upgrade then of any kind}
% The strategy of {\nsu} avoid the attack in Figure~\ref{fig:fs-conc-attack}.  The
% reason for that is that |shared_ref| would not be \emph{automatically} upgraded
% by thread A. Consequently, command |upgrade share_ref medium| succeeds and so
% both threads would always run to completion, producing no leaks.


% Local Variables:
% TeX-master: "main.tex"
% TeX-command-default: "Make"
% End: