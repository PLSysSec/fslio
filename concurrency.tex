\section{Concurrency}
\label{sec:conc}

\concept{extending \lioafs{}}
In this section, we consider flow-sensitive references in the presence of
concurrency.
%
Concretely, we extend our sequential \liofs{} and \lioafs{} calculi with
threads and a new terminal, |forkLIO|, for dynamically creating new threads, as
in the concurrent version of LIO~\cite{stefan:addressing-covert}.
%
Intuitively, this concurrent calculus \lioconc{} simply defines a scheduler over
sequential threads, such that taking a step in the concurrent calculus amounts
to taking a step in a sequential thread and context switching to a different
thread.
%
For brevity, we restrict our discussion in this section to the case where the
underlying sequential calculus is \lioafs{}, since this calculus extends
\liofs.

\begin{figure}
\small
\begin{code}
t    ::= cdots  | forkLIO t | noToLabeled
\end{code}
\begin{mathpar}
\inferrule[forkLIO]
{ }
{
|conf c (E[forkLIO t]) .-->  conf c (E[return ()])|
}
\and
\inferrule[withRefs-Opt]
{ 
|v = refs ((addrs(v1)) and (addrs(v2)))|\\
|conf c (E[withRefs v t] --> conf c' (E[t'])|
}
{
|conf c (E[withRefs v1 (withRefs v2 t)]) --> conf c' (E[t'])|
}
\and
\inferrule[T-step]
{
|c = (lcurr, mI, mS)|\\
|conf c (withRefs v t) -->  conf c' t'|\\
|c' = (lcurr', mI', mS')|\\
|v' = refs(dom mS')|
}
{
|tconf mI mS (thread lcurr v t, k2, ...) --> tconf mI' mS' (k2, ..., thread lcurr' v' t')|
}
\and
\inferrule[T-stuck]
{ }
{
|tconf mI mS (thread lcurr v diverge, k2, ...) --> tconf mI mS (k2, ...)|
}
\and
\inferrule[T-done]
{ }
{
|tconf mI mS (thread lcurr v v', k2, ...) --> tconf mI mS (k2, ...)|
}

\and
\inferrule[T-fork]
{
|c = (lcurr, mI, mS)|\\
|conf c (withRefs v t) .-.->  conf c' t''|\\
|c' = (lcurr', mI', mS')|\\
|v' = refs(dom mS')|\\
|knew = thread lcurr' v' t'|
}
{
|tconf mI mS (thread lcurr v t, k2, ...) --> tconf mI' mS' (k2, ..., thread lcurr' v' t'', knew)|
}
\end{mathpar}
  \caption{Semantics for \lioconc{}, parametric in the flow-sensitivity policy,
  i.e., with and without auto-upgrade\label{fig:sos:conc}.}
\end{figure}

\concept{con progs}
Figure~\ref{fig:sos:conc} shows our extended concurrent calculus, \lioconc{}.
%
A concurrent program configuration has the form |tconf mI mS (k1, k2,...)|,
where |mI| and |mS| are respectively the flow-insensitive and flow-sensitive
stores shared by all the threads |k1, k2, ...| in the program.
%
Since the memory stores are global, a thread |k| is simply a tuple
encapsulating the current label of the thread |lcurr|, the term under
evaluation |t|, and a bag of references |v| the thread may access, i.e.,
|k=thread lcurr v t|.

\concept{step}
The reduction rules for concurrent programs are mostly standard.
%
Rule \ruleref{T-step} specifies that if the first thread in the thread pool
takes a step in \lioafs{}, the whole concurrent program takes a step, moving
the thread to the end of the pool.
%
We note that the thread term |t| executed with its stored current label
|lcurr|, and a subset of the flow-sensitive memory store, by wrapping it in
|withRefs|.
%
While the use of |withRefs| makes the extension straight forward, one
peculiarity arises: since \ruleref{T-step} always wraps the thread term |t|
with |withRefs|, if |t| does not reduce in one step to a value, and instead
reduces to |t'|, the next time the thread is scheduled we will superfluously
wrap |withRefs t'| with yet another |withRefs|---thus preventing the thread
from making progress!
%
To address this we extend the calculus with rule \ruleref{withRefs-Opt} that
collapses nested |withRefs| blocks.\footnote{
This change also requires modifying \ruleref{withRefs-Ctx} to not trigger when
the term being evaluated is a |withRefs| term.
}
 
\concept{done/stuck}
Rules \ruleref{T-done} and \ruleref{T-stuck} specify that once a thread term
has reduced to a value or got stuck, which is represented by |diverge|, the
scheduler removes it from the thread pool and schedules the next thread.

\concept{fork}
%
As shown in Figure~\ref{fig:sos:conc}, to allow for dynamic thread creation, we
extend \lioafs{}'s terms with |forkLIO|, and add a new reduction rule that sends
an event to the scheduler, specifying the term to execute in a new
thread.\footnote{
In fact the reduction rule for \lioafs{} must be changed to account for events.
However, since |fork| is the only event in our system, we treat |-->| as
implicitly carrying an empty event.
}
%
Rule \ruleref{T-fork} describes the corresponding scheduler rule, triggered
when a |fork(t')| event is received.
%
Here, we create a new thread |knew| whose current label |lcurr'| and partition
of the store, i.e., bag of references |v'|, is the same as that of the parent
thread; the term evaluated in the newly created thread is provided in the
event: |t'|.
%
Subsequently, we add the new thread to the thread pool.

\concept{no toLabeled}
The final modification in extending \lioafs{} to \lioconc{} shown in
Figure~\ref{fig:sos:conc} is the removal of |toLabeled| from the underlying
calculus.
%
As described in~\cite{stefan:addressing-covert}, we must remove |toLabeled| to
guarantee termination-sensitive non-interference.
%
Importantly, however |forkLIO| with synchronization primitives (e.g.,
flow-insensitive labeled MVars, as discussed
in~\cite{stefan:addressing-covert}) can be used to provide functionality
equivalent to that of |toLabeled|;
%
we omit synchronization primitives from \lioconc{} for simplicity of
exposition.\footnote{
Our approach to safely adding flow-sensitive references can directly
be applied to other objects.
%
Extending \lioconc{} to provide flow-sensitive labeled MVars of follows
identically, and similar to the embedding of \liofs{} into \lio{}, we can embed
\lioconc{} with flow-sensitive MVars into \lioconc.
}

\concept{still have attack}
The flow-sensitive attack in Figure~\ref{fig:fs-attack} relied on |toLabeled|
to restore the current label and thus allow for the leak of |high| data to be
observed in a |low| context.
%
A natural question, given that we remove |toLabeled|, is whether we can use the
naive flow-sensitive reference semantics of Section~\ref{sec:flow-sensitive}
for concurrent LIO.
%
As shown by the attack code in Figure~\ref{fig:fs-conc-attack}, in which we use
|forkLIO| instead of |toLabeled| to address a potential label creep, the
fundamental problem remains: the label on the reference label is not protected!
%
\begin{figure}
\small
\begin{code}
leakRef :: LIORefTCB space Bool -> LIO Bool
leakRef href = do
  tmp   <- newRef space low ()
  forkLIO $ do  h <- readRef space href
                when h (writeRef space tmp ())
  delay
  return (labelOfR space tmp == high)
\end{code}
\cut{$}
\caption{Attack in concurrent LIO with naive flow-sensitive reference
extension. We omit subscripts on reference terminals for clarity.
\label{fig:fs-conc-attack}}
\end{figure}
%
This precisely motivated our principled approach of extending \lioafs{} to a
concurrent setting as opposed to extending concurrent LIO with flow-sensitive
references.

% Local Variables:
% TeX-master: "main.tex"
% TeX-command-default: "Make"
% End:
